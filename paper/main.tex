\documentclass[letterpaper]{article}
\usepackage{natbib,alifeconf}
\usepackage{graphicx,amsmath,booktabs,xcolor,multirow}

\title{Digital Life: Satisfying Seven Biological Criteria\\Through Functional Analogy and Criterion-Ablation}
\author{Anonymous}

\begin{document}
\maketitle

% ============================================================================
\begin{abstract}
We present a testable integration of all seven textbook biological criteria
for life---cellular organization, metabolism, homeostasis, growth,
reproduction, response to stimuli, and evolution---as functionally
interdependent computational processes within a single artificial life system.
Existing systems implement at most a subset of these criteria, often as
independent modules or static proxies that can be removed without measurable
system degradation.
Our hybrid swarm-organism architecture implements each criterion as a dynamic
process satisfying three conditions---sustained resource consumption,
measurable degradation upon removal, and feedback coupling with other
criteria---which we term \emph{functional analogy}.
A criterion-ablation experiment ($n$=30 per condition, seeds held out from
calibration) demonstrates that disabling any single criterion causes
statistically significant population decline (Mann-Whitney $U$,
Holm-Bonferroni corrected, all $p \leq 0.016$), with Cliff's $\delta$
ranging from 0.32 to 1.00.
Pairwise ablations reveal sub-additive interactions consistent with
shared failure pathways, confirming that criteria damage overlapping
subsystems rather than operating independently.
A proxy control comparison shows that metabolism implementations of
differing complexity produce qualitatively distinct population dynamics,
ruling out tautological criterion definitions.
The strongest single-ablation effects arise from disabling reproduction
($\Delta$=--89.3\%), response to stimuli ($\Delta$=--88.4\%), and metabolism
($\Delta$=--84.3\%), confirming that these criteria function as necessary,
interdependent components of organismal viability rather than decorative
labels.
\end{abstract}

% ============================================================================
\section{Introduction}

What distinguishes a living system from a merely complex one?
Biology textbooks identify seven criteria---cellular organization, metabolism,
homeostasis, growth and development, reproduction, response to stimuli, and
evolution \citep{urry_2020_campbell}---but artificial life (ALife) research has
struggled to integrate all seven into a single computational system.
Most existing platforms implement a subset: evolutionary dynamics without
metabolism \citep{ofria_2004_avida}, pattern formation without reproduction
\citep{chan_2019_lenia}, or boundary self-organization without evolution
\citep{plantec_2023_flow}.
Where criteria are nominally present, they often function as \emph{simplified
proxies}---static parameters or independent modules whose removal has no
measurable effect on system behavior.

We argue that a meaningful computational model of life requires more than
feature checklists.
Each criterion must function as a \emph{functional analogy} of its biological
counterpart, satisfying three conditions:
\begin{enumerate}
  \item \textbf{Dynamic process}: the criterion requires sustained resource
    consumption at every timestep, not a static lookup.
  \item \textbf{Measurable degradation}: ablating the criterion causes
    statistically significant decline in organism viability.
  \item \textbf{Feedback coupling}: the criterion forms at least one feedback
    loop with another criterion, precluding independent-module implementations.
\end{enumerate}
This definition operationalizes the intuition behind autopoiesis
\citep{maturana_1980_autopoiesis}, the chemoton model's integration
of metabolic, boundary, and information subsystems
\citep{ganti_2003_principles}, and minimal-life frameworks
\citep{ruizmirazo_2004_universal} in a form amenable to experimental
falsification.

We adopt a \emph{weak ALife} stance: our system is a functional model of life,
not a claim that the organisms are alive.
The contribution is methodological---demonstrating that all seven criteria
\emph{can} be integrated as interdependent processes and that their necessity
can be rigorously tested.

This paper makes four contributions:
\begin{enumerate}
  \item A hybrid swarm-organism architecture integrating all seven biological
    criteria as functionally interdependent processes.
  \item An operational definition of \emph{functional analogy} with three
    falsifiable conditions.
  \item A \emph{criterion-ablation} methodology demonstrating each criterion's
    functional necessity via controlled experiments with multiple-comparison
    correction.
  \item Pairwise ablation and proxy control experiments characterizing
    criterion interactions and ruling out tautological definitions.
\end{enumerate}

% ============================================================================
\section{Related Work}

Artificial life research has produced diverse computational substrates, each
excelling along different axes of biological fidelity.
We organize the landscape by methodological approach rather than chronology.

\paragraph{Evolutionary platforms.}
Tierra \citep{ray_1991_approach} and Avida \citep{ofria_2004_avida} pioneered
self-replicating digital organisms with mutation and natural selection,
achieving strong evolutionary dynamics (Level~5 on our rubric).
However, organisms lack spatial bodies, metabolic processes, and boundary
maintenance---criteria 1--4 receive minimal or no implementation.
Polyworld \citep{yaeger_1994_computational} adds neural-network-driven behavior
and simple energy budgets, though its metabolism operates as a dynamic
single-resource process rather than a multi-step network.

\paragraph{Continuous and particle-based systems.}
Lenia \citep{chan_2019_lenia} demonstrates emergent lifelike patterns in
continuous cellular automata, exhibiting coherent spatial organization and
growth.
Flow-Lenia \citep{plantec_2023_flow} extends this with mass conservation,
achieving stronger boundary maintenance.
ALIEN \citep{heinemann_2008_alien} provides a GPU-accelerated particle
simulator with typed cells supporting self-replicating structures,
achieving multi-process interaction on several criteria.
While ALIEN demonstrates broad coverage, no existing system combines
multi-step metabolism with active homeostatic regulation.

\paragraph{Chemistry-inspired systems.}
Coralai \citep{barbieux_2024_coralai} combines multi-agent neural cellular
automata with energy dynamics, implementing rudimentary metabolism and
spatial organization.
It represents a step toward metabolic integration but does not achieve
the feedback coupling between metabolism and other criteria that our
framework requires.

\paragraph{Theoretical foundations.}
The autopoiesis framework \citep{maturana_1980_autopoiesis,
mcmullin_2004_thirty} emphasizes self-producing boundaries as the
minimal criterion for life, which our boundary-maintenance mechanism
operationalizes.
NASA's working definition---``a self-sustaining chemical system capable of
Darwinian evolution'' \citep{joyce_1994_foreword}---bundles metabolism and
evolution but does not individuate the remaining five criteria.
\citet{ruizmirazo_2004_universal} argue for a richer set of minimal
conditions, closer to our seven-criteria framework.
The open-ended evolution research program \citep{bedau_2000_open,
taylor_2016_openended} provides metrics for evolutionary richness that complement
our criterion-ablation approach.

Table~\ref{tab:comparison} summarizes how existing systems score on each
criterion using a five-level rubric (1=absent, 5=self-maintaining/emergent).

\begin{table*}[t]
\centering
\caption{Literature comparison: seven biological criteria scored on a
five-level rubric (1=no feature, 2=static parameter, 3=dynamic single process,
4=multi-process interaction, 5=self-maintaining/emergent).
Bold indicates scores $\geq$4.}
\label{tab:comparison}
\small
\begin{tabular}{l@{\hskip 6pt}c@{\hskip 6pt}c@{\hskip 6pt}c@{\hskip 6pt}c@{\hskip 6pt}c@{\hskip 6pt}c@{\hskip 6pt}c@{\hskip 6pt}c}
\toprule
System & Cell.Org & Metab & Homeo & Growth & Reprod & Response & Evol & Total \\
\midrule
% Scores verified via literature review; justifications in paper/literature_scores.md
Polyworld  & 2 & 3 & 1 & 1 & 3 & \textbf{4} & \textbf{4} & 18 \\
Avida      & 2 & 3 & 1 & 2 & \textbf{4} & 3 & \textbf{5} & 20 \\
Lenia      & 3 & 1 & 2 & 2 & 2 & 3 & 2 & 15 \\
ALIEN      & \textbf{4} & 3 & 2 & 3 & \textbf{4} & \textbf{4} & \textbf{4} & 24 \\
Flow-Lenia & 3 & 3 & 3 & 3 & 3 & 3 & 3 & 21 \\
Coralai    & 3 & 3 & 2 & 3 & 3 & 3 & 3 & 20 \\
\midrule
\textbf{Ours}$^\dagger$ & \textbf{4} & \textbf{4} & \textbf{4} & 3 & \textbf{4} & \textbf{4} & 3 & \textbf{26} \\
\bottomrule
\multicolumn{9}{l}{\footnotesize $^\dagger$Self-assessment; scores may differ under external evaluation.}
\end{tabular}
\end{table*}

While ALIEN achieves Level~4 on four criteria, it lacks multi-step metabolism
(Level~3) and active homeostasis (Level~2).
Our system reaches $\geq$4 on five of seven criteria, with the highest total
score (26) across all systems surveyed.
We note that these scores are self-assessed; independent evaluation may adjust
individual ratings.

% ============================================================================
\section{System Design}

\subsection{Architecture Overview}

The system implements a hybrid two-layer architecture (Figure~\ref{fig:arch}).
The outer layer is a continuous toroidal 2D environment (100$\times$100 world
units) containing a diffusing resource field.
The inner layer consists of 10--50 \emph{organisms}, each composed of 10--50
\emph{swarm agents} that collectively maintain the organism's spatial boundary.

\begin{figure*}[t]
\centering
\includegraphics[width=\textwidth]{figures/fig_architecture.pdf}
\caption{Two-layer architecture. Each organism comprises swarm agents maintaining
a spatial boundary, a neural-network controller, a graph-based metabolic network,
and a variable-length genome encoding all seven criteria. Organisms inhabit a
continuous toroidal environment with a diffusing resource field.}
\label{fig:arch}
\end{figure*}

Each organism maintains the following runtime state: boundary integrity
($b \in [0,1]$), metabolic state (energy $e$, resource $r$, waste $w$),
internal state vector for homeostatic regulation, a neural-network controller,
a genetically encoded metabolic network, age, generation counter, and maturity
level.

\subsection{Seven Criteria Implementation}

Table~\ref{tab:criteria} maps each biological criterion to its computational
implementation, ablation toggle, and feedback partners.

\begin{table}[t]
\centering
\caption{Mapping of seven biological criteria to computational processes.
Each criterion satisfies the three functional-analogy conditions.}
\label{tab:criteria}
\small
\begin{tabular}{@{}p{1.4cm}p{2.8cm}p{2.5cm}@{}}
\toprule
Criterion & Process & Feedback \\
\midrule
Cell.\ Org. & Swarm agents maintain boundary; decays without energy & Metab, Homeo \\
Metabolism & Graph network transforms resources to energy; waste accumulates & Cell.\ Org., Homeo \\
Homeostasis & NN regulates internal state vector each step & Metab, Response \\
Growth & Maturation from seed to full capacity & Metab, Reprod \\
Reproduction & Division when metabolically ready; energy cost; offspring from seed & Metab, Evol \\
Response & NN processes sensory input $\rightarrow$ velocity delta & Homeo, Metab \\
Evolution & Mutation during reproduction; differential survival & Reprod, all \\
\bottomrule
\end{tabular}
\end{table}

\paragraph{Cellular organization.}
Swarm agents collectively define an organism's spatial extent.
Boundary integrity $b$ decays each step at a base rate modulated by energy
deficit and waste pressure:
$\Delta b_{\text{decay}} = -r_b \cdot (1 + s_e \cdot (1 - e) + s_w \cdot w)$,
where $r_b = 0.02$ is the base decay rate, $s_e = 0.5$ and $s_w = 0.3$ are
scaling factors.
Repair occurs proportionally to available energy:
$\Delta b_{\text{repair}} = r_r \cdot e \cdot (1 - s_p \cdot w)$,
with repair rate $r_r = 0.15$ and waste penalty $s_p = 0.4$.
When $b$ falls below a collapse threshold ($b < 0.1$), the organism dies.

\paragraph{Metabolism.}
Each organism possesses a genetically encoded graph-based metabolic network
with 2--4 catalytic nodes and directed edges.
The genome segment (16 floats) is decoded via sigmoid mapping:
node count $= \text{round}(\sigma(g_0) \cdot 2 + 2)$,
catalytic efficiency $= \sigma(g_{2+i}) \cdot 0.9 + 0.1 \in [0.1, 1.0]$,
edge existence determined by $|g_j| > 0.3$, and
conversion efficiency $= \sigma(g_{13}) \cdot 0.7 + 0.3 \in [0.3, 1.0]$.
External resources enter at a designated entry node, flow through the graph
with per-edge transfer efficiency in $[0.7, 1.0]$, and exit as energy.
Waste accumulates as a byproduct proportional to throughput.

\paragraph{Homeostasis.}
A feedforward neural network (8 inputs $\rightarrow$ 16 hidden with tanh
$\rightarrow$ 4 outputs with tanh; 212 weights) processes sensory inputs
(normalized position $x,y$; velocity $v_x, v_y$; internal state
$s_0, s_1, s_2$; neighbor density) and produces velocity
adjustments and internal-state deltas.
The internal state vector enables adaptive regulation: organisms that maintain
internal variables within viable ranges survive longer.

\paragraph{Growth and development.}
Organisms begin as minimal seeds (maturity $m = 0$) and develop toward full
capacity ($m = 1$) over time.
Maturation gates metabolic throughput and reproductive readiness, ensuring
organisms must develop before they can reproduce.

\paragraph{Reproduction.}
When energy exceeds $e_{\min} = 0.7$ and boundary integrity exceeds
$b_{\min} = 0.5$, an organism may divide.
The parent pays an energy cost ($c_r = 0.3$), and the offspring inherits a
(possibly mutated) copy of the genome, starting as a seed.
Child agents spawn within a radius of the parent's center of mass.

\paragraph{Response to stimuli.}
The neural-network controller processes a local sensory field each timestep,
producing velocity deltas that govern agent movement.
Disabling response freezes agents' velocity adjustments, preventing adaptive
resource seeking.

\paragraph{Evolution.}
During reproduction, offspring genomes undergo point mutations
(rate $= 0.01$ per gene, scale $= 0.1$), reset mutations
(rate $= 0.001$), and scale mutations (rate $= 0.005$,
factor $\in [0.8, 1.2]$).
All gene values are clamped to $[-5, 5]$.
This produces heritable variation subject to differential survival.

\subsection{Genome Encoding}

The genome is a variable-length vector of 256 floats organized into seven
segments: neural-network weights (212), metabolic network (16), homeostasis
parameters (8), developmental program (8), reproduction parameters (4),
sensory parameters (4), and evolution parameters (4).
All criteria are encoded from initialization; segments are activated as
features are enabled.

% ============================================================================
\section{Criterion-Ablation Experiment}

This experiment tests whether each of the seven criteria is functionally
necessary for organism viability, as predicted by the functional-analogy
framework.

\subsection{Protocol}

The system provides seven boolean ablation toggles, one per criterion
(e.g., \texttt{enable\_metabolism = false}).
For each of the seven criteria, we disable that criterion while keeping all
others active, and compare the resulting population dynamics against the
fully enabled baseline (``normal'' condition).
This yields eight conditions: one normal baseline and seven single-criterion
ablations.

\subsection{Data Separation}

To prevent overfitting of thresholds, we separate data into:
\begin{itemize}
  \item \textbf{Calibration set}: Seeds 0--99, used during development for
    parameter tuning and threshold selection.
  \item \textbf{Test set}: Seeds 100--129 ($n$=30), held out until final
    evaluation. All reported results use this set exclusively.
\end{itemize}

Calibration confirmed that both metabolism engines produce viable populations
(Toy: $\bar{x}$=328.1, Graph: $\bar{x}$=291.8 alive at step 2000).
All final experiments use the Graph metabolism engine.

\subsection{Simulation Parameters}

Each simulation runs for 2000 timesteps with population sampled every 50
steps.
The environment is a 100$\times$100 toroidal grid with 30 initial organisms,
each comprising 25 swarm agents.
The primary outcome metric is alive organism count at step 2000.

\subsection{Statistical Design}

For each ablation condition, we test the one-sided hypothesis:
\[
H_1: \text{alive\_count}_{\text{normal}} > \text{alive\_count}_{\text{ablated}}
\]
using the Mann-Whitney $U$ test \citep{mann_1947_test}, appropriate for
non-normal count data.
We apply Holm-Bonferroni correction \citep{holm_1979_simple} for seven
simultaneous comparisons at $\alpha = 0.05$.
Effect sizes are reported as both Cohen's $d$ \citep{cohen_1988_statistical}
and Cliff's $\delta$ \citep{cliff_1993_dominance}, the latter being more
appropriate for non-normal distributions.
We additionally report the area under the alive-count curve (AUC) and
median organism lifespan as secondary outcome measures.

% ============================================================================
\section{Results}

All seven criterion ablations produce statistically significant population
decline compared to the normal baseline (Table~\ref{tab:ablation}).

\begin{table}[t]
\centering
\caption{Criterion-ablation results ($n$=30 per condition). Normal baseline
mean: 293.1 (median: 294, IQR: 282--310). All $p$-values Holm-Bonferroni
corrected. $^{***}p<0.001$, $^{*}p<0.05$.}
\label{tab:ablation}
\small
\begin{tabular}{@{}lrrrrrr@{}}
\toprule
Condition & Mean & $\Delta$\% & $d$ & Cliff's $\delta$ & $p_{\text{corr}}$ & Sig. \\
\midrule
No Reproduction & 31.3 & $-$89.3 & 18.06 & 1.00 & $<$0.001 & $^{***}$ \\
No Response     & 34.1 & $-$88.4 & 17.82 & 1.00 & $<$0.001 & $^{***}$ \\
No Metabolism   & 46.0 & $-$84.3 & 17.24 & 1.00 & $<$0.001 & $^{***}$ \\
No Homeostasis  & 68.3 & $-$76.7 & 4.94  & 1.00 & $<$0.001 & $^{***}$ \\
No Boundary     & 121.8 & $-$58.5 & 4.36 & 1.00 & $<$0.001 & $^{***}$ \\
No Growth       & 185.6 & $-$36.7 & 5.34 & 1.00 & $<$0.001 & $^{***}$ \\
No Evolution    & 278.3 & $-$5.1  & 0.57 & 0.32 & 0.016 & $^{*}$ \\
\bottomrule
\end{tabular}
\end{table}

Three ablations cause near-total population collapse ($>$84\% decline):
reproduction, response to stimuli, and metabolism.
These criteria form the core viability loop---without energy production,
adaptive movement, or population renewal, organisms cannot sustain themselves.

Figure~\ref{fig:timeseries} shows population trajectories across all
conditions.
The normal condition (black) stabilizes around 293 organisms by step~1900.
Metabolic ablation (orange) causes rapid collapse within the first 200 steps,
as organisms cannot produce energy to maintain boundaries.
Reproduction ablation (blue) produces a slower but equally terminal decline,
as the initial population ages and dies without replacement.
Evolution ablation (purple) shows the weakest effect ($d$=0.57), with
populations remaining viable but slightly smaller than normal---consistent
with evolution operating as an optimization process rather than a survival
necessity at these timescales.

\begin{figure*}[t]
\centering
\includegraphics[width=\textwidth]{figures/fig_timeseries.pdf}
\caption{Population dynamics under criterion ablation. Lines show mean alive
count across 30 seeds (100--129); shaded bands show $\pm$1 SEM. Normal
baseline (thick black) stabilizes near 293 organisms by step~1900. Removing reproduction,
response, or metabolism causes $>$84\% population collapse. Evolution ablation
shows a modest 5\% decline ($d$=0.57), consistent with optimization rather
than short-term survival necessity.}
\label{fig:timeseries}
\end{figure*}

\paragraph{Functional analogy verification.}
For each criterion, all three conditions are satisfied:
(a)~each consumes resources per step (energy for boundary repair, metabolic
computation, NN evaluation);
(b)~ablation causes significant degradation (Table~\ref{tab:ablation}); and
(c)~feedback loops are observable (e.g., metabolism~$\leftrightarrow$~boundary:
energy funds repair, boundary collapse stops metabolism).
Thus, each criterion qualifies as a functional analogy, not a simplified proxy.

\subsection{Proxy Control Comparison}

To test whether criterion ablation merely reflects tautological
definitions, we compare three metabolism implementations on the same seeds
($n$=30): \textbf{Counter} (minimal single-step conversion, no waste),
\textbf{Toy} (single-step with waste dynamics), and \textbf{Graph} (full
multi-step network with catalytic nodes).
All three satisfy ``dynamic resource consumption,'' yet produce
qualitatively different dynamics (Figure~\ref{fig:proxy}).
Graph metabolism supports the highest genome diversity (7.58 vs.\ 5.69 for
Toy) despite sustaining fewer organisms (293 vs.\ 322 for Toy, 374 for
Counter), indicating that metabolic complexity imposes greater selective
pressure.
This confirms that the specific implementation---not merely the presence---of
a criterion shapes system behavior, ruling out tautological definitions.

\begin{figure}[t]
\centering
\IfFileExists{figures/fig_proxy.pdf}%
  {\includegraphics[width=\columnwidth]{figures/fig_proxy.pdf}}%
  {\fbox{\parbox{\dimexpr\columnwidth-2\fboxsep-2\fboxrule}{%
    \centering\vspace{2em}[fig\_proxy.pdf --- to be generated]\vspace{2em}}}}
\caption{Proxy control comparison. Three metabolism implementations of
increasing complexity on the same seeds ($n$=30). Graph metabolism sustains
fewer organisms but higher genome diversity and waste dynamics, demonstrating
that metabolic complexity produces qualitatively distinct ecological dynamics
rather than simply increasing population counts.}
\label{fig:proxy}
\end{figure}

\subsection{Pairwise Ablations and Interdependence}

To test for interaction effects beyond individual necessity, we disable
pairs of criteria and compute synergy:
$\text{synergy}_{A,B} = \Delta_{A \cup B} - (\Delta_A + \Delta_B)$.
Table~\ref{tab:pairwise} reports scores for six pairs.
All show sub-additive synergy (negative), indicating \emph{shared failure
pathways}: individual ablations already collapse populations to near
their floor ($\sim$30--50 organisms), leaving no room for additive effects.
This ceiling effect reveals that criteria damage overlapping subsystems.
The (metabolism, homeostasis) pair exemplifies the shared pathway:
disabling metabolism eliminates energy production, starving boundary
repair ($\Delta b_{\text{repair}} \propto e$); disabling homeostasis
degrades adaptive behavior, accelerating waste accumulation, which
amplifies boundary decay via the waste-pressure term ($s_w \cdot w$
in $\Delta b_{\text{decay}}$).
Both routes converge on boundary failure, explaining why
$\Delta_{AB} = 249.9$ falls far below the additive expectation of 472.0.

\begin{table}[t]
\centering
\caption{Pairwise ablation synergy scores ($n$=30 per condition, Graph
metabolism). Negative synergy indicates sub-additive interaction (shared
failure pathways). Baseline $\bar{x}$=293.1, consistent with
Table~\ref{tab:ablation}.}
\label{tab:pairwise}
\small
\begin{tabular}{@{}lrrrr@{}}
\toprule
Pair & $\Delta_{AB}$ & Expected & Synergy & Ratio \\
\midrule
(Metab, Homeo)    & 249.9 & 472.0 & $-$222.0 & 0.53 \\
(Metab, Response)  & 247.7 & 506.1 & $-$258.4 & 0.49 \\
(Reprod, Growth)   & 261.9 & 369.4 & $-$107.5 & 0.71 \\
(Boundary, Homeo)  & 181.9 & 396.2 & $-$214.3 & 0.46 \\
(Response, Homeo)  & 256.4 & 483.9 & $-$227.5 & 0.53 \\
(Reprod, Evol)     & 261.9 & 276.7 & $-$14.9  & 0.95 \\
\bottomrule
\multicolumn{5}{l}{\footnotesize Expected = $\Delta_A + \Delta_B$ (additive null model); Ratio = $\Delta_{AB}$ / Expected.}
\end{tabular}
\end{table}

\paragraph{Growth--reproduction confound.}
The (reproduction, growth) pair confirms: $\Delta_{AB} \approx
\Delta_{\text{reproduction}}$ = 261.9, so growth's effect is fully mediated
through reproduction gating.
This clarifies growth's role as an upstream prerequisite for reproductive
competence rather than an independent viability mechanism.

\subsection{Evolution Strengthening}

The modest single-ablation effect of evolution ($d$=0.57, Cliff's
$\delta$=0.32) reflects the 2000-step simulation horizon.
To demonstrate evolution's contribution at longer timescales, we run two
additional experiments:

\paragraph{Long run.}
Extending simulations to 10,000 steps ($n$=30) yields $d$=1.43, Cliff's
$\delta$=0.72 ($p < 0.001$), a large effect compared to the modest
2000-step result.
Normal populations reach $\bar{x}$=358.1 organisms versus 327.1 for
no-evolution ($\Delta$=$-$8.7\%), confirming that evolutionary adaptation
accumulates across generations.

\paragraph{Environmental shift.}
At step 2,500 of a 5,000-step simulation, resource regeneration rate is
halved (from 0.01 to 0.005 per cell per step).
Evolved populations recover more effectively ($\bar{x}$=384.6 vs.\ 361.4,
$d$=1.01, Cliff's $\delta$=0.57, $p < 0.001$), demonstrating that
evolution enables adaptive response to environmental perturbation.

\begin{figure}[t]
\centering
\IfFileExists{figures/fig_evolution.pdf}%
  {\includegraphics[width=\columnwidth]{figures/fig_evolution.pdf}}%
  {\fbox{\parbox{\dimexpr\columnwidth-2\fboxsep-2\fboxrule}{%
    \centering\vspace{2em}[fig\_evolution.pdf --- to be generated]\vspace{2em}}}}
\caption{Evolution strengthening. Top: 10,000-step long run shows
increasing divergence between normal and no-evolution conditions.
Bottom: environmental shift at step 2,500 (dashed line) demonstrates
evolved populations' superior recovery.
Lines show mean across 30 seeds; shaded bands $\pm$1 SEM.}
\label{fig:evolution}
\end{figure}

\subsection{Homeostatic Regulation}

Figure~\ref{fig:homeostasis} shows population-mean internal state
trajectories for Normal versus No Homeostasis conditions.
Under normal operation, the neural-network controller actively regulates
internal state variable~$s_0$, maintaining it near 0.99 throughout
the simulation (Panel~A, solid).
When homeostasis is disabled, $s_0$ decays monotonically due to the
passive decay term ($h_{\text{decay}} = 0.01$ per step), and the
population exhibits high variance as organisms lack the ability to
counteract perturbations (Panel~A, dashed; variance shown in Panel~B).
This divergence confirms that the NN-driven homeostatic mechanism
constitutes an active regulatory process---not a static parameter---that
satisfies the functional-analogy requirement of sustained resource
consumption with measurable degradation upon removal.

\begin{figure}[t]
\centering
\IfFileExists{figures/fig_homeostasis.pdf}%
  {\includegraphics[width=\columnwidth]{figures/fig_homeostasis.pdf}}%
  {\fbox{\parbox{\dimexpr\columnwidth-2\fboxsep-2\fboxrule}{%
    \centering\vspace{2em}[fig\_homeostasis.pdf --- to be generated]\vspace{2em}}}}
\caption{Homeostatic regulation trajectories ($n$=30). (A)~Mean internal
state variable~$s_0$ over time: Normal condition (solid) maintains
regulation near 0.99, while No Homeostasis (dashed) shows passive decay.
(B)~Population-level standard deviation of $s_0$: disabled homeostasis
produces higher inter-organism variance. Shaded bands show $\pm$1 SEM.}
\label{fig:homeostasis}
\end{figure}

% ============================================================================
\section{Discussion}

\paragraph{Criterion interdependence.}
Single ablations reveal a hierarchy: reproduction, response, and metabolism
form an essential triad ($>$84\% decline), while homeostasis and boundary
occupy a middle tier ($\sim$58--77\%).
Pairwise ablations show uniformly sub-additive interactions
(Table~\ref{tab:pairwise}), consistent with \emph{shared failure pathways}
rather than independent modules---criteria converge on overlapping viability
subsystems.
The proxy control comparison further demonstrates that the specific
implementation of a criterion shapes ecological dynamics (graph metabolism
produces higher diversity but lower populations than simpler alternatives),
ruling out tautological definitions.

\paragraph{Evolution at longer timescales.}
The modest 2000-step evolution effect ($d$=0.57) grows to $d$=1.43 at
10,000 steps and $d$=1.01 under environmental perturbation, confirming
that evolutionary adaptation accumulates across generations.

\subsection{Limitations}

\textbf{Cellular organization} is tracked via a scalar boundary-integrity
variable rather than emergent spatial cohesion among swarm agents;
an explicit spatial boundary model would provide a stronger functional
analogy to biological membranes.
\textbf{Growth} uses a maturation toggle that functions primarily as a
reproduction gate rather than a full developmental program;
a morphogenetic model would strengthen this criterion's
functional-analogy claim.
\textbf{Evolution} reaches $d$=1.43 at 10,000 steps but demonstrating
open-ended dynamics would require $10^5$+ steps with novelty metrics
\citep{bedau_2000_open}.
\textbf{Scale} is limited to $\sim$300 organisms on a single machine;
larger populations might reveal emergent ecological phenomena.
We adopt a \textbf{weak ALife} stance: this is a functional model, not
a claim that digital organisms are alive.

% ============================================================================
\section{Conclusion}

We presented a testable integration of all seven textbook biological criteria
as functionally interdependent processes within a single artificial life system,
verified through controlled criterion-ablation, pairwise interaction, and proxy
control experiments.
The functional-analogy framework---requiring dynamic operation, measurable
degradation upon removal, and feedback coupling---provides a rigorous
standard distinguishing genuine criteria implementations from simplified
proxies.

Our results demonstrate that no criterion is decorative: removing any one
causes statistically significant population decline ($p < 0.016$, Holm-Bonferroni
corrected), with Cliff's $\delta$ ranging from 0.32 to 1.00.
Pairwise ablations further reveal shared failure pathways---sub-additive
interactions consistent with criteria converging on overlapping viability
subsystems rather than operating independently.

Future work will pursue three directions:
(1)~a richer developmental program replacing the current growth toggle;
(2)~scaling to larger populations to investigate emergent ecological phenomena
and open-ended evolution metrics \citep{bedau_2000_open,
taylor_2016_openended}; and
(3)~systematic environmental perturbation studies to characterize adaptive
capacity across evolutionary timescales.

Code and data will be made available upon acceptance at an anonymous repository.

\bibliographystyle{apalike}
\bibliography{references}

\end{document}
