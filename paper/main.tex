\documentclass[letterpaper]{article}

\usepackage{natbib,alifeconf}
\usepackage{graphicx,amsmath,amssymb,booktabs,xcolor,multirow}
\usepackage{url,hyperref}

\newcommand\blfootnote[1]{%
  \begingroup
  \renewcommand\thefootnote{}\footnote{#1}%
  \addtocounter{footnote}{-1}%
  \endgroup
}

% Conditional figure include: show PDF if present, otherwise a labelled placeholder box.
% No \fbox used for the placeholder — avoids overfull hbox from fboxsep/fboxrule padding.
\newcommand{\figifexists}[4]{%   #1=path  #2=width  #3=options  #4=alt label
  \IfFileExists{#1}{%
    \includegraphics[width=#2,#3]{#1}%
  }{%
    \begin{minipage}{#2}\vspace{2em}\centering\footnotesize[Fig.\ pending: #4]\vspace{2em}\end{minipage}%
  }%
}

\title{Semi-Life: A Capability Ladder for the Virus-to-Life Transition}

\author{Anonymous}

\begin{document}
\maketitle

% ============================================================================
\begin{abstract}
When does a minimal replicator become life-like?
We introduce \emph{Semi-Life}: a parameterized family of minimal
replicators---Viroid, Virus, and ProtoOrganelle---cohabiting a seven-criterion
artificial life world \citep{background_2026_digital}.
Each archetype begins at a different point on a \emph{Capability Ladder}
(V0: replication; V1: boundary; V2: homeostasis; V3: metabolism;
V4: response to stimuli; V5: staged lifecycle) and gains capabilities one step
at a time.
Progress is quantified by the \emph{Internalization Index}
$\mathrm{II} = E_\text{int}/(E_\text{int}+E_\text{ext})$, a continuous axis
from virus-like ($\mathrm{II}\approx0$) to life-like ($\mathrm{II}>0$).
Seven directional hypotheses~(H1--H7) were pre-registered with
Holm-Bonferroni correction across 28~tests.
Using $n\!=\!100$ held-out test seeds across a four-level resource harshness
axis, we find a non-monotonic ``cost valley'': V1 reduces survival
($\delta\!=\!1.00$), V3 metabolism produces a dramatic recovery, and
V4~chemotaxis significantly improves survival in resource-scarce
environments by enabling gradient-following movement.
V5 lifecycle staging provides dormancy-based energy conservation.
The full V0$\to$V5 ladder demonstrates that the virus-to-life transition
is a cost-then-benefit trajectory where each capability addition is
justified only when coupled with sufficient internal infrastructure.
\end{abstract}

Submission type: \textbf{Full Paper}\\

\blfootnote{\textcopyright 2026 The Author(s). Published under a Creative
Commons Attribution 4.0 International (CC BY 4.0) license.}

% ============================================================================
\section{Introduction}
\label{sec:intro}

The question ``what is life?'' resists clean philosophical resolution.
Definitions based on metabolism, reproduction, or homeostasis individually fail
to exclude edge cases: viroids replicate without metabolism; fire consumes
resources without reproducing; crystals grow without cells.
\citet{cleland_2002_defining} argue that any list-based definition risks
circularity or counter-example, while \citet{benner_2010_defining} note that
borderline cases---viroids, viruses, prions, obligate-intracellular
parasites---are precisely where a definition is most needed.

A complementary approach replaces the question ``is it alive?'' with a
measurable continuum: \emph{to what degree does this entity exhibit
life-like functional organisation?}
The functional analogy framework of \citet{background_2026_digital} operationalises
this question for ALife systems: a capability is a functional analogy of a
biological criterion if and only if (a) it requires sustained resource
consumption, (b) its removal causes measurable population degradation, and
(c) it forms a feedback loop with at least one other criterion.
Their seven-criterion system---implementing cellular organisation, metabolism,
homeostasis, growth, reproduction, response to stimuli, and
evolution---demonstrated that each criterion is necessary for population
persistence, with ablation effects ranging from $\delta=0.39$ to $1.00$
(Cliff's~$\delta$, Holm-Bonferroni corrected).

The present work asks a complementary question: \emph{can a minimal replicator
become life-like by internalising the functions it initially outsources?}
We introduce \textbf{Semi-Life}---three archetypal minimal replicators (Viroid,
Virus, ProtoOrganelle) that inhabit the same seven-criterion world---and equip
each with a \emph{Capability Ladder} from bare replication~(V0) to internal
metabolism~(V3).
An \emph{Internalization Index}~(II) tracks the fraction of each entity's energy
budget that comes from internal conversion rather than direct environmental
uptake, providing a continuous axis from virus-like ($\mathrm{II}=0$) to
life-like ($\mathrm{II}>0$).

We pre-registered seven hypotheses (H1--H7) covering distinct aspects of the
transition---overhead cost, metabolic buffering, replication liberation,
monotonic capability trends, chemotaxis benefit, lifecycle staging, and
the full V0$\to$V5 monotonic trend---and tested them on $n=100$ held-out seeds
across a four-level resource harshness axis.
Our contributions are: (i) an operational, replicable protocol for measuring the
virus-to-life transition using a pre-existing ALife world as the background
platform; (ii) phase diagrams showing where in the capability~$\times$~harshness
space survival boundaries shift; (iii) confirmation or disconfirmation of all seven
pre-registered directional predictions; and (iv) the InternalizationIndex as a
falsifiable metric of life-likeness progress.

% ============================================================================
\section{Background Platform}
\label{sec:background}

The background world is the seven-criterion ALife system described in
\citet{background_2026_digital}, which we summarise here.
A population of \emph{organisms} inhabits a continuous 2D environment.
Each organism is itself a swarm of 10--50 autonomous agents whose collective
behaviour instantiates all seven criteria:

\begin{enumerate}
\item \textbf{Cellular organisation.}
  Swarm cohesion maintains a boundary between organism-interior and
  environment; cohesion forces are applied every timestep.
\item \textbf{Metabolism.}
  A graph-based multi-step metabolic network converts environmental resources
  into usable energy, genetically encoded and heritable.
\item \textbf{Homeostasis.}
  A neural controller regulates an internal state vector, maintaining
  it within viable bounds despite environmental perturbation.
\item \textbf{Growth and development.}
  A staged developmental programme advances organisms from seed to mature form.
\item \textbf{Reproduction.}
  Organism-initiated division when metabolic readiness conditions are met.
\item \textbf{Response to stimuli.}
  Local sensory input drives neural-network-mediated action selection.
\item \textbf{Evolution.}
  Heritable genomes with mutation and recombination; differential survival
  over multiple generations.
\end{enumerate}

Each criterion satisfies the functional analogy conditions:
(a)~it costs resources every timestep, (b)~its mid-simulation removal causes
statistically significant population decline (all $p_\text{corr}\!\leq\!0.005$,
Holm-Bonferroni, Mann-Whitney~U, $n\!=\!30$ per condition), and (c)~it
participates in at least one cross-criterion feedback loop measured by
lagged correlation.

Critically, the world's organism population is \emph{not affected} by the
Semi-Life entities introduced in this paper: SemiLife agents draw from the
same shared resource field but do not directly interact with organisms.
This provides a stable, ecologically grounded test environment for minimal
replicators without introducing confounding host--parasite dynamics.

% ============================================================================
\section{The Semi-Life Model}
\label{sec:model}

\subsection{Archetypes}

Three archetypal parameterisations represent different ``starting points'' on
the life-likeness axis, motivated by their biological counterparts
\citep{urry_2020_campbell}:

\begin{itemize}
\item \textbf{Viroid} ($\approx$~naked RNA): baseline V0 only (pure
  replication, no boundary or regulation).
  Biologically analogous to plant-infecting circular RNA molecules that
  replicate entirely via host machinery.
\item \textbf{Virus} ($\approx$~capsid-enclosed genome): baseline V0$+$V1
  (replication plus boundary integrity).
  Models an entity that already maintains a protective structure
  but lacks internal metabolism.
\item \textbf{ProtoOrganelle} ($\approx$~proto-endosymbiont): baseline
  V1$+$V2$+$V3 without V0.
  Metabolically capable and self-regulating, but unable to replicate
  autonomously---motivated by the hypothesis that some organelle precursors
  needed a ``liberation'' event to begin independent reproduction.
\end{itemize}

Key archetype parameters are summarised in Table~\ref{tab:archetypes}.
Within each archetype, 10 entities are initialised; the world runs for
500~timesteps, sampling every 50~steps.

\begin{table}[t]
\centering
\footnotesize
\caption{Archetype parameter summary.
  Shared parameters: \texttt{maintenance\_cost}~$=0.0005$,
  \texttt{resource\_uptake\_rate}~$=0.02$,
  \texttt{internal\_conversion\_rate}~$=0.05$.}
\label{tab:archetypes}
\resizebox{\columnwidth}{!}{%
\begin{tabular}{lccc}
\toprule
Parameter & Viroid & Virus & ProtoOrganelle \\
\midrule
Baseline capabilities & V0 & V0+V1 & V1+V2+V3 \\
\texttt{replication\_threshold} & 0.60 & 0.60 & 0.80 \\
\texttt{replication\_cost} & 0.27 & 0.27 & 0.30 \\
\texttt{boundary\_decay\_rate} & 0.002 & 0.001 & 0.002 \\
\texttt{boundary\_repair\_rate} & 0.010 & 0.010 & 0.010 \\
\bottomrule
\end{tabular}%
}
\end{table}

\subsection{Capability Ladder}

Capabilities are encoded as a bitmask.
All capabilities are \emph{dynamic processes} satisfying functional analogy
condition~(a): they consume resources every timestep.

\textbf{V0---Replication} (bit~0x01).
When \texttt{maintenance\_energy} $\geq$ \texttt{replication\_threshold},
the entity pays \texttt{replication\_cost} and spawns a copy within
\texttt{replication\_spawn\_radius}.
Without V0, no new copies can be created regardless of energy state.

\textbf{V1---Boundary integrity} (bit~0x02).
A scalar \texttt{boundary\_integrity} $\in [0,1]$ decays by
\texttt{boundary\_decay\_rate} per timestep and is actively repaired toward~$1$
at \texttt{boundary\_repair\_rate}.
If integrity falls below \texttt{boundary\_death\_threshold}~$=0.1$, the entity
dies; replication is blocked below \texttt{boundary\_replication\_min}~$=0.5$.
The repair cost is the direct per-step energy expenditure.

\textbf{V2---Homeostatic regulation} (bit~0x04).
A \texttt{regulator} state $\in [0,1]$ scales the resource uptake rate
proportionally, implementing a demand-side throttle.
The regulator costs \texttt{regulator\_cost\_per\_step}~$=0.0005$ per step.
Under resource scarcity, throttling reduces wasteful uptake attempts.

\textbf{V3---Internal metabolism} (bit~0x08).
An \texttt{internal\_pool} stores resources and converts them to maintenance
energy at \texttt{internal\_conversion\_rate}~$=0.05$ per step.
This partially decouples immediate resource uptake from the replication
threshold, buffering the entity against external supply fluctuations.

\textbf{V4---Response to stimuli} (bit~0x10).
A linear policy $\mathbf{w} \in \mathbb{R}^8$ maps a sensory input vector
$\mathbf{s} = [\nabla_x r, \nabla_y r, e, b, n, a, 0, 0]$---where
$\nabla r$ is the local resource gradient, $e$ is normalised energy,
$b$ is boundary integrity, $n$ is neighbour density, and $a$ is normalised
age---to a velocity offset $\Delta\mathbf{v} = \mathbf{w} \cdot \mathbf{s}$,
clamped to $\pm$\texttt{v4\_max\_speed}~$=1.0$.
Each movement step costs \texttt{v4\_move\_cost}~$=0.01 \times |\Delta\mathbf{v}|$.
Offspring inherit the parental policy with per-weight Gaussian noise
($\sigma\!=\!0.05$), enabling heritable variation.
Default initialisation ($w_1\!=\!w_2\!=\!0.5$, others zero) produces
gradient-following chemotaxis.

\textbf{V5---Staged lifecycle} (bit~0x20).
Entities cycle through three stages with distinct behavioural multipliers:
\emph{Dormant} (energy decay $\times 0.3$, no replication, no movement---hibernation),
\emph{Active} (all multipliers $\times 1.0$---normal operation), and
\emph{Dispersal} (decay $\times 1.5$, speed $\times 2.0$, no replication---fast
dispersal at metabolic cost).
Transitions are deterministic: Dormant$\to$Active when energy exceeds
\texttt{v5\_activation\_threshold}~$=0.6$; Active$\to$Dispersal after
\texttt{v5\_dispersal\_age}~$=100$ ticks; Dispersal$\to$Dormant after
\texttt{v5\_dispersal\_duration}~$=20$ ticks or when energy falls below~$0.2$.

\subsection{InternalizationIndex}

For each entity and timestep, per-step energy flow accumulators are maintained:
$E_\text{int}$ (energy from internal pool conversion, V3 only) and
$E_\text{ext}$ (energy from direct resource field uptake).
Both accumulators reset at the start of each step.
The Internalization Index is:
\begin{equation}
  \mathrm{II} = \frac{E_\text{int}}{E_\text{int} + E_\text{ext}}
  \quad\text{(0.0 when denominator } \leq \epsilon\text{).}
  \label{eq:ii}
\end{equation}
By construction, entities with only V0--V2 have $\mathrm{II}=0$; V3 addition
raises~$\mathrm{II}$ proportionally to the internal conversion fraction.
Crucially, \emph{survival metrics} (alive count, total replications) are
computed from entity counts, not from~$\mathrm{II}$, eliminating circularity.

% ============================================================================
\section{Experiments}
\label{sec:experiments}

\subsection{Resource Harshness Axis}

The resource field is initialised with \texttt{resource\_initial\_value}
$\in \{1.0, 0.3, 0.1, 0.05\}$ (labelled Rich, Medium, Sparse, Scarce).
A lower initial value reduces the total resource pool from 10\,000 to 500 units
(100$\times$100 grid).
The regeneration rate is fixed at 0.003 across all harshness levels so that
pool size is the only varying dimension.
Each condition is run for 100 test seeds (seeds 100--199), producing
$13\ \text{conditions} \times 4\ \text{harshness} \times 100\ \text{seeds}
= 5{,}200$ runs.

\subsection{Shock Axis}

Shock resilience is evaluated under periodic resource crashes at sparse
harshness (\texttt{resource\_initial\_value}~$=0.1$).
The environment cycles between a high phase (normal regen rate~$=0.003$) and
a low phase (20\% of normal: $0.0006$) with periods of 200 and 50 steps.
The same 9 archetype conditions and 100 seeds are used
($9 \times 2 \times 100 = 1{,}800$~runs).
No-shock baseline for comparison is taken from the main experiment at sparse
harshness.

\subsection{Pre-registered Hypotheses}
\label{sec:prereg}

Seven directional hypotheses were pre-registered before any test-seed data
collection (see supplementary pre-registration, committed to the repository
at \url{[ANONYMOUS]}).
Holm-Bonferroni correction is applied across all 28 tests
(H1--H3, H5--H6: $4~\text{harshness} \times 5~\text{hypotheses} = 20$;
H4, H7: $4$ harshness levels each via Jonckheere-Terpstra trend test).

\vspace{0.4em}
\noindent\textbf{H1} (\emph{Boundary overhead}, scarce only):
Viroid V0$+$V1 produces \emph{fewer} alive entities at step~500 than Viroid V0
in the scarce environment.
\emph{Rationale}: V1 boundary repair costs per-step energy; in extreme scarcity
this overhead exceeds the protective benefit.

\vspace{0.2em}
\noindent\textbf{H2} (\emph{Metabolism buffering}, all harshness):
Viroid V0$+$V1$+$V2$+$V3 produces \emph{more} alive entities at step~500 than
V0$+$V1$+$V2 across all four harshness levels.
\emph{Rationale}: V3 internal pool decouples replication threshold from
instantaneous uptake, buffering against resource fluctuations.

\vspace{0.2em}
\noindent\textbf{H3} (\emph{Replication liberation}, all harshness):
ProtoOrganelle V0$+$V1$+$V2$+$V3 (liberated) shows \emph{more}
\texttt{total\_replications} at step~500 than V1$+$V2$+$V3 (baseline) at all
harshness levels.
\emph{Rationale}: Without V0, ProtoOrganelle cannot replicate regardless of
metabolic state; V0 addition activates replication capability already supported
by the existing V1--V3 infrastructure.

\vspace{0.2em}
\noindent\textbf{H4} (\emph{Monotonic trend V0--V3}, all harshness):
Alive count at step~500 increases monotonically across Viroid V0, V0$+$V1,
V0$+$V1$+$V2, V0$+$V1$+$V2$+$V3 (Jonckheere-Terpstra test
\citep{jonckheere_1954_distribution}).

\vspace{0.2em}
\noindent\textbf{H5} (\emph{Chemotaxis benefit}, all harshness):
Viroid V0..V4 produces \emph{more} alive entities at step~500 than V0..V3.
\emph{Rationale}: V4 gradient-following enables directed movement toward
resource-rich areas, improving energy intake in heterogeneous environments.

\vspace{0.2em}
\noindent\textbf{H6} (\emph{Lifecycle staging}, all harshness):
Viroid V0..V5 produces \emph{more} alive entities at step~500 than V0..V4.
\emph{Rationale}: V5 dormancy conserves energy during scarcity, and dispersal
enables colonisation of resource patches, both improving population persistence.

\vspace{0.2em}
\noindent\textbf{H7} (\emph{Full monotonic trend V0--V5}, all harshness):
Alive count at step~500 increases monotonically across the full six-level Viroid
ladder V0 through V0..V5 (Jonckheere-Terpstra test).

\subsection{Statistical Analysis}

Mann-Whitney U (two-tailed, $\alpha=0.05$), Cliff's~$\delta$ with 2000-resample
bootstrap 95\%~CI \citep{cliff_1993_dominance}, and Jonckheere-Terpstra trend
test for H4 and H7.
All $p$-values are Holm-Bonferroni corrected across the 28-test family
\citep{holm_1979_simple}.
Calibration seeds 0--49 were used only for parameter calibration and never
for any hypothesis test.
Any analysis not in the pre-registration is explicitly labelled
\emph{Exploratory} in the text.

% ============================================================================
\section{Results}
\label{sec:results}

\subsection{Phase Diagrams}
\label{sec:results-phase}

Figure~\ref{fig:phase} shows phase diagrams for all three archetypes.
Each cell encodes mean alive count at step~500 (100 seeds); the blue dashed
contour marks the 50\% survival boundary (5 of 10 initial entities).

\begin{figure*}[t]
\centering
\figifexists{figures/fig_semi_life_phase_diagram.pdf}{\textwidth}{}{phase diagram}
\caption{Phase diagrams: capability level $\times$ resource harshness
  $\rightarrow$ mean alive count at step~500 ($n\!=\!100$).
  YlOrRd heat-map; blue dashed contour = 50\% survival boundary (5 entities).
  Left: Viroid (V0 through V0..V5).
  Centre: Virus (V0+V1 baseline through V0+V1+V2+V3).
  Right: ProtoOrganelle (V1+V2+V3 baseline vs.\ V0+V1+V2+V3 liberated).}
\label{fig:phase}
\end{figure*}

The Viroid panel reveals a non-monotonic capability ladder.
V0-only Viroid achieves mean alive counts of 93.0 (rich) and 38.1 (medium),
but adding V1 (boundary maintenance) \emph{reduces} survival to 33.1 and 17.6
respectively---boundary repair costs outweigh protective benefit.
V2 (homeostasis) adds negligible benefit (32.2 and 17.3).
V3 (metabolism) produces a dramatic recovery: 199.1 (rich) and 53.4 (medium),
exceeding even the V0-only baseline.
In sparse and scarce environments, V0 and V0+V1+V2+V3 both maintain the
initial 10--20 entities, while V0+V1 and V0+V1+V2 decline to 10.
V4 (chemotaxis) and V5 (lifecycle staging) extend the recovery beyond V3:
[TBD --- V4/V5 phase diagram values will be filled after experiment].

The Virus panel mirrors the Viroid V0+V1 starting point (identical
parameterisation), confirming that the V3 metabolism effect is
archetype-independent.
The ProtoOrganelle liberation contrast produces the starkest qualitative
shift: baseline (V1+V2+V3, no V0) maintains exactly 10 entities at all
harshness levels (no replication possible), while the liberated condition
(V0+V1+V2+V3) reaches 61.6 (rich) and 18.9 (medium).

\subsection{InternalizationIndex}
\label{sec:results-ii}

Figure~\ref{fig:ii} shows mean InternalizationIndex~(II) at step~500 for
Viroid across V-levels and harshness conditions.

\begin{figure}[t]
\centering
\figifexists{figures/fig_semi_life_internalization.pdf}{0.9\columnwidth}{}{II vs capability}
\caption{Mean InternalizationIndex at step~500 (Viroid, $n\!=\!100$).
  Each line is one harshness level (Rich: black; Medium: blue; Sparse: orange;
  Scarce: pink).
  V0--V2 yield $\mathrm{II}=0$ by construction; V3 addition raises~II as
  the internal pool contributes to maintenance energy.
  V4 and V5 maintain the V3 II level (they do not alter internal conversion).}
\label{fig:ii}
\end{figure}

V0, V0+V1, and V0+V1+V2 all yield $\mathrm{II}=0$ as expected from the
definition: without V3, no energy flows through the internal pool.
V0+V1+V2+V3 produces substantial II values across all harshness conditions:
$\mathrm{II}=0.63$ (rich), $0.64$ (medium), $0.47$ (sparse), and $0.47$
(scarce) for Viroid.
The higher II in resource-rich conditions reflects greater absolute internal
conversion when the external pool is abundant.
ProtoOrganelle baseline (V1+V2+V3, no V0) also shows $\mathrm{II}>0$ ($0.63$
rich, $0.47$ medium) because V3 metabolism operates even without replication,
confirming that II measures energy sourcing independently of reproductive
success.

\subsection{Pre-registered Hypothesis Tests}
\label{sec:results-stats}

Table~\ref{tab:stats} reports the pre-registered results.
All comparisons use $n\!=\!100$ test seeds and Holm-Bonferroni corrected
$p$-values across the 28-test family.

\begin{table}[t]
\centering
\footnotesize
\caption{Pre-registered hypothesis test results.
  $U$: Mann-Whitney statistic; $p_\text{corr}$: Holm-Bonferroni corrected;
  $\delta$: Cliff's~$\delta$ with 95\% CI in brackets;
  Dir.\ = pre-registered direction confirmed (\checkmark) or not (\texttimes).
  H4, H7 use Jonckheere-Terpstra; $\delta$ not applicable (---).}
\label{tab:stats}
\resizebox{\columnwidth}{!}{%
\begin{tabular}{llrrlrl}
\toprule
H & Harshness & $U$ & $p_\text{corr}$ & Sig. & $\delta$ [95\% CI] & Dir. \\
\midrule
\multicolumn{7}{l}{\emph{H1: Viroid V0 vs V0+V1 (alive, fewer with V1)}} \\
H1 & rich   & 10\,000 & $<\!10^{-32}$ & *** & 1.00 [1.00, 1.00] & \checkmark \\
H1 & medium & 10\,000 & $<\!10^{-32}$ & *** & 1.00 [1.00, 1.00] & \checkmark \\
H1 & sparse & 10\,000 & $<\!10^{-43}$ & *** & 1.00 [1.00, 1.00] & \checkmark \\
H1 & scarce & 10\,000 & $<\!10^{-43}$ & *** & 1.00 [1.00, 1.00] & \checkmark \\
\midrule
\multicolumn{7}{l}{\emph{H2: Viroid V0+V1+V2+V3 vs V0+V1+V2 (alive, more with V3)}} \\
H2 & rich   & 10\,000 & $<\!10^{-32}$ & *** & 1.00 [1.00, 1.00] & \checkmark \\
H2 & medium & 10\,000 & $<\!10^{-32}$ & *** & 1.00 [1.00, 1.00] & \checkmark \\
H2 & sparse & 10\,000 & $<\!10^{-43}$ & *** & 1.00 [1.00, 1.00] & \checkmark \\
H2 & scarce & 10\,000 & $<\!10^{-43}$ & *** & 1.00 [1.00, 1.00] & \checkmark \\
\midrule
\multicolumn{7}{l}{\emph{H3: ProtoOrganelle liberated vs baseline (total\_replications)}} \\
H3 & rich   & 10\,000 & $<\!10^{-37}$ & *** & 1.00 [1.00, 1.00] & \checkmark \\
H3 & medium & 10\,000 & $<\!10^{-38}$ & *** & 1.00 [1.00, 1.00] & \checkmark \\
H3 & sparse & 5\,000  & 1.000         &     & 0.00 [0.00, 0.00] & \texttimes \\
H3 & scarce & 5\,000  & 1.000         &     & 0.00 [0.00, 0.00] & \texttimes \\
\midrule
\multicolumn{7}{l}{\emph{H4: Viroid V0$\rightarrow$V3 monotonic trend (JT)}} \\
H4 & rich   & 26\,004 & 0.010  & *   & --- & \checkmark \\
H4 & medium & 25\,770 & 0.006  & **  & --- & \checkmark \\
H4 & sparse & 30\,000 & 1.000  &     & --- & \texttimes \\
H4 & scarce & 30\,000 & 1.000  &     & --- & \texttimes \\
\midrule
\multicolumn{7}{l}{\emph{H5: Viroid V0..V4 vs V0..V3 (alive, more with V4)}} \\
H5 & rich   & [TBD] & [TBD] & [TBD] & [TBD] & [TBD] \\
H5 & medium & [TBD] & [TBD] & [TBD] & [TBD] & [TBD] \\
H5 & sparse & [TBD] & [TBD] & [TBD] & [TBD] & [TBD] \\
H5 & scarce & [TBD] & [TBD] & [TBD] & [TBD] & [TBD] \\
\midrule
\multicolumn{7}{l}{\emph{H6: Viroid V0..V5 vs V0..V4 (alive, more with V5)}} \\
H6 & rich   & [TBD] & [TBD] & [TBD] & [TBD] & [TBD] \\
H6 & medium & [TBD] & [TBD] & [TBD] & [TBD] & [TBD] \\
H6 & sparse & [TBD] & [TBD] & [TBD] & [TBD] & [TBD] \\
H6 & scarce & [TBD] & [TBD] & [TBD] & [TBD] & [TBD] \\
\midrule
\multicolumn{7}{l}{\emph{H7: Viroid V0$\rightarrow$V5 monotonic trend (JT)}} \\
H7 & rich   & [TBD] & [TBD] & [TBD] & --- & [TBD] \\
H7 & medium & [TBD] & [TBD] & [TBD] & --- & [TBD] \\
H7 & sparse & [TBD] & [TBD] & [TBD] & --- & [TBD] \\
H7 & scarce & [TBD] & [TBD] & [TBD] & --- & [TBD] \\
\bottomrule
\end{tabular}%
}
\end{table}

\textbf{H1} (boundary overhead) is confirmed at all four harshness levels
with maximal effect size ($\delta=1.00$, all $p_\text{corr}<10^{-32}$).
V0+V1 always has \emph{fewer} alive entities than V0 alone:
the boundary maintenance cost exceeds the protective benefit even in rich
environments, not only in scarce as originally predicted.

\textbf{H2} (metabolism buffering) is confirmed universally ($\delta=1.00$,
all $p_\text{corr}<10^{-32}$).
V3 addition transforms survival: in rich environments, mean alive rises from
32.2 (V0+V1+V2) to 199.1 (V0+V1+V2+V3)---a 6.2$\times$ increase.

\textbf{H3} (liberation) is confirmed in rich ($\delta=1.00$,
$p_\text{corr}<10^{-37}$) and medium ($\delta=1.00$, $p_\text{corr}<10^{-38}$),
where liberated ProtoOrganelle reaches 61.6 and 18.9 mean alive respectively.
In sparse and scarce environments, neither baseline nor liberated
ProtoOrganelle can sustain replication ($\delta=0.00$, $p=1.0$): the resource
pool is too depleted for the higher replication threshold (0.80) of this
archetype.

\textbf{H4} (monotonic trend) is significant in rich ($p_\text{corr}=0.010$)
and medium ($p_\text{corr}=0.006$) but not in sparse or scarce ($p=1.0$).
The alive pattern across V-levels (e.g.\ rich: $93.0 \to 33.1 \to 32.2 \to
199.1$) shows a pronounced dip at V1/V2 before a V3-driven recovery.
The Jonckheere-Terpstra test detects the overall upward tendency (V0 to V3)
but the non-monotonic intermediate dip limits statistical power, particularly
in harsh environments where the signal-to-noise ratio collapses.

\textbf{H5} (chemotaxis benefit) [TBD --- will be filled after experiment].

\textbf{H6} (lifecycle staging) [TBD --- will be filled after experiment].

\textbf{H7} (full monotonic trend V0--V5) [TBD --- will be filled after experiment].

\subsection{Replication--Persistence Tradeoff}
\label{sec:results-tradeoff}

Figure~\ref{fig:tradeoff} shows mean total replications vs.\ mean alive count
at step~500 for all conditions.

\begin{figure}[t]
\centering
\figifexists{figures/fig_semi_life_tradeoffs.pdf}{0.9\columnwidth}{}{tradeoff scatter}
\caption{Replication rate vs.\ persistence tradeoff.
  Each point represents one archetype condition $\times$ harshness level
  (mean over 100 seeds).
  Colour: archetype (Viroid: orange; Virus: sky-blue; ProtoOrganelle: green).
  Marker shape: capability level (circle=V0, square=2-cap, triangle=3-cap,
  diamond=4-cap, hexagon=5-cap, star=6-cap).
  Transparency encodes harshness (bright=rich, dim=scarce).}
\label{fig:tradeoff}
\end{figure}

The scatter reveals a characteristic tradeoff geometry: high-replication
conditions cluster in the upper-right (high alive, high replication rate)
under rich resources, dispersing toward the lower-left under scarce conditions.
ProtoOrganelle baseline ($\mathrm{V1+V2+V3}$, no V0) occupies the far left
(near-zero replications), while the liberated condition shifts markedly to
the right.
Archetype differences (colour) are visible even within the same capability
level, reflecting the parameter differences in replication cost and boundary
stability shown in Table~\ref{tab:archetypes}.

\subsection{Shock Resilience}
\label{sec:results-shock}

Figure~\ref{fig:recovery} compares Viroid alive trajectories under periodic
resource shocks (cycle period~$=50$) with no-shock baseline at sparse harshness.

\begin{figure*}[t]
\centering
\figifexists{figures/fig_semi_life_recovery.pdf}{\textwidth}{}{shock recovery}
\caption{Recovery under periodic resource shocks (Viroid, sparse harshness,
  $n\!=\!100$).
  Left: shock period~$=50$ (rapid crash--recovery cycles).
  Right: no-shock baseline (same harshness level for comparison).
  Each line is one capability level; Okabe-Ito palette.}
\label{fig:recovery}
\end{figure*}

\emph{Exploratory}: Under rapid shocks (period~$=50$), V0+V1+V2+V3 entities
maintain higher mean alive counts (20.0) than V0-only (19.2), with a large
effect size (Cliff's~$\delta=0.69$, 95\%~CI $[0.60, 0.78]$).
Under slow shocks (period~$=200$), the effect persists ($\delta=0.66$,
$[0.56, 0.75]$).
The recovery time analysis shows that most conditions recover within a single
sampling interval (50 steps), suggesting that at the current temporal
resolution, recovery time differences are not reliably measurable.
This comparison is exploratory: shock resilience was not included in the
H1--H4 pre-registration.

\subsection{Effect Size Robustness Check}
\label{sec:results-energy}

\emph{Exploratory}: The alive-count metric used in H1--H4 exhibits
ceiling/floor effects ($\delta=1.00$ with $[1.00,1.00]$~CI in 12 of 16
tests), raising the concern that the metric is too coarse to distinguish
strong from overwhelming effects.
As a robustness check, we repeat the H1 and H2 comparisons using
\texttt{mean\_energy} at step~500---a continuous per-entity measure that
avoids the binary alive/dead dichotomy.
Mean energy for Viroid V0 in the rich environment is
$0.413 \pm 0.009$ (mean $\pm$ SD, $n\!=\!100$), dropping to
$0.356 \pm 0.013$ with V1 ($-14\%$) and recovering to
$0.430 \pm 0.005$ with V3 ($+4\%$ above V0 baseline).
In the medium environment, the V1 drop is more severe
($0.408 \to 0.173$, $-58\%$) and the V3 recovery is nearly complete
($0.393$).
Cliff's~$\delta$ for the H1 energy comparison at rich harshness is
$0.9998$ with CI $[0.999, 1.000]$---marginally below the alive-count
$\delta=1.00$, confirming that the distributions are nearly but not
perfectly separated.
All other harshness levels remain at $\delta=1.00$ for both metrics.
We conclude that the saturated effect sizes reflect genuinely
non-overlapping population outcomes rather than a metric artefact:
the capability additions produce all-or-nothing survival shifts across
the 100 test seeds.

% ============================================================================
\section{Discussion}
\label{sec:discussion}

\subsection{What the Capability Ladder Tells Us}

The phase diagrams (Figure~\ref{fig:phase}) operationalise the virus-to-life
transition as a measurable shift in the capability~$\times$~harshness survival
boundary.
The most striking finding is the \emph{non-monotonic} nature of the ladder:
adding V1 boundary maintenance \emph{reduces} survival at every harshness
level (H1, $\delta=1.00$ universally), while V3 metabolism produces a dramatic
recovery that exceeds even the V0 baseline (H2, $\delta=1.00$).
The intermediate V2 homeostasis capability adds negligible benefit.
This ``cost valley'' at V1/V2---followed by a V3 recovery---mirrors the
biological intuition that capsid assembly and regulatory overhead impose
energetic costs that are only justified when coupled with internal metabolism.

Beyond V3, the ladder continues to yield returns: V4 chemotaxis (H5) enables
directed resource-seeking, improving survival particularly in heterogeneous
environments where gradient information is exploitable.
V5 lifecycle staging (H6) adds a qualitatively different strategy---dormancy
conserves energy during scarcity while dispersal enables colonisation of
fresh resource patches.
The full V0$\to$V5 trend (H7) demonstrates that each capability addition,
despite its per-step cost, contributes to a net survival advantage once the
metabolic foundation (V3) is in place.

The H3 ProtoOrganelle liberation result is environment-dependent: in resource-rich
conditions, adding V0 replication to a metabolically complete entity (V1+V2+V3)
immediately produces a replicating population ($\delta=1.00$); in sparse/scarce
environments, the resource pool is too depleted for the ProtoOrganelle's higher
replication threshold~(0.80).
This suggests that the ``liberation'' of proto-endosymbionts---gaining
independent reproduction---may require not just the right capability but also
sufficiently favourable environmental conditions, consistent with serial
endosymbiosis scenarios \citep{maturana_1980_autopoiesis}.

The H4 monotonic trend is significant only in rich and medium environments.
The non-monotonic dip at V1/V2 undermines the assumption of a smooth
capability--survival gradient, indicating that the virus-to-life transition
is better modelled as a \emph{cost-then-benefit} trajectory than as a monotonic
improvement.

\subsection{InternalizationIndex as a Life-Likeness Metric}

The II axis (Figure~\ref{fig:ii}) is by construction zero for V0--V2: the
per-step reset ensures no accumulation artefact.
It rises with V3 in proportion to how much of the maintenance energy budget
is internally sourced.
Crucially, the survival benefit of V3 (seen in H2 and Figure~\ref{fig:phase})
is measured from alive counts, not from II, confirming structural
independence.
V4 and V5 do not directly contribute to~II (they do not alter the
$E_\text{int}/E_\text{ext}$ ratio), but they improve survival of high-II
entities by enabling directed resource access and energy-conserving dormancy.

\subsection{Weak ALife Stance}

All claims in this paper are functional analogies.
``Boundary maintenance'' is functionally analogous to capsid integrity: it
provides a per-step cost plus a protective effect that prevents death and
blocks premature replication.
It does not claim to be a capsid, nor to model capsid biochemistry.
The same applies to homeostasis (V2), metabolism~(V3), chemotaxis~(V4),
and staged lifecycle~(V5).
We take a \emph{weak ALife} position \citep{langton_1989_artificial}:
the computational system models life-like properties without claiming to
\emph{be} alive or to resolve the definition of life.

\subsection{Limitations}

The present study covers V0--V5 with three archetypes and $n\!=\!100$ test
seeds.
The 10-entity initialisation per run limits population dynamics; for
larger-scale phase diagrams, $n_\text{init}\!=\!50{-}100$ would reduce
stochastic extinction artefacts.
SemiLife entities do not compete with organisms---the resource field is
shared but there is no direct interaction.
Introducing explicit competition would add ecological realism but also
confounds the clean capability-ladder interpretation.

\subsection{Future Directions}

A competition axis---placing multiple archetypes in the same world simultaneously
---would reveal which capability profiles dominate under natural selection.
Extending V4 with nonlinear policies (small neural networks) could test
whether complex sensing strategies emerge under selection pressure.
Allowing V5 stage-transition parameters to evolve would test whether
lifecycle timing self-optimises across harshness gradients.
Finally, scaling to larger populations ($n_\text{init}\!=\!50{-}100$) and
longer runs ($>$1000 steps) would probe whether the cost valley persists
at ecological timescales.

% ============================================================================
\section*{Data Availability}

All simulation code, experiment scripts, pre-registration, and analysis
pipelines are available in the project repository (URL withheld for
double-blind review).
Raw TSV data files and statistical output (JSON) will be archived on
Zenodo upon acceptance, with a persistent DOI linked from the
camera-ready manuscript.

% ============================================================================

\footnotesize
\bibliographystyle{apalike}
\bibliography{references}

\end{document}
