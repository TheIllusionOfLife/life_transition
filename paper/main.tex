\documentclass[letterpaper]{article}

\usepackage{natbib,alifeconf}
\usepackage{graphicx,amsmath,amssymb,booktabs,xcolor,multirow}
\usepackage{url,hyperref}

\newcommand\blfootnote[1]{%
  \begingroup
  \renewcommand\thefootnote{}\footnote{#1}%
  \addtocounter{footnote}{-1}%
  \endgroup
}

% Conditional figure include: show PDF if present, otherwise a labelled placeholder box.
% No \fbox used for the placeholder — avoids overfull hbox from fboxsep/fboxrule padding.
\newcommand{\figifexists}[4]{%   #1=path  #2=width  #3=options  #4=alt label
  \IfFileExists{#1}{%
    \includegraphics[width=#2,#3]{#1}%
  }{%
    \begin{minipage}{#2}\vspace{2em}\centering\footnotesize[Fig.\ pending: #4]\vspace{2em}\end{minipage}%
  }%
}

\title{Semi-Life: A Capability Ladder for the Virus-to-Life Transition}

\author{Anonymous}

\begin{document}
\maketitle

% ============================================================================
\begin{abstract}
When does a minimal replicator become life-like?
We introduce \emph{Semi-Life}: a parameterized family of minimal
replicators---Viroid, Virus, and ProtoOrganelle---cohabiting a seven-criterion
artificial life world \citep{background_2026_digital}.
Each archetype begins at a different point on a \emph{Capability Ladder}
(V0: replication; V1: boundary maintenance; V2: homeostatic regulation;
V3: internal metabolism) and gains capabilities one step at a time.
Progress is quantified by the \emph{Internalization Index}
$\mathrm{II} = E_\text{int}/(E_\text{int}+E_\text{ext})$, a continuous axis
from virus-like ($\mathrm{II}\approx0$, all energy from the environment)
to life-like ($\mathrm{II}>0$, self-sustained energy conversion).
We pre-registered four directional hypotheses~(H1--H4) covering boundary
overhead costs, metabolism-driven persistence, replication liberation, and
monotonic capability--survival trends across a four-level resource harshness
axis, applying Holm-Bonferroni correction across all 16 tests.
Using $n\!=\!100$ held-out test seeds (never used during parameterisation),
we find that internal metabolism~(V3) provides the strongest survival benefit
in resource-scarce environments (H2), boundary maintenance~(V1) imposes
measurable overhead costs in the scarce regime consistent with H1,
and adding replication capability~(V0) to a ProtoOrganelle liberates
population growth consistent with H3.
Phase diagrams over the capability~$\times$~harshness plane reveal where
the survival boundary shifts, making the virus-to-life transition
operationally measurable without requiring a host organism.
\end{abstract}

Submission type: \textbf{Full Paper}\\

\blfootnote{\textcopyright 2026 [AUTHORS' NAMES]. Published under a Creative
Commons Attribution 4.0 International (CC BY 4.0) license.}

% ============================================================================
\section{Introduction}
\label{sec:intro}

The question ``what is life?'' resists clean philosophical resolution.
Definitions based on metabolism, reproduction, or homeostasis individually fail
to exclude edge cases: viroids replicate without metabolism; fire consumes
resources without reproducing; crystals grow without cells.
\citet{cleland_2002_defining} argue that any list-based definition risks
circularity or counter-example, while \citet{benner_2010_defining} note that
borderline cases---viroids, viruses, prions, obligate-intracellular
parasites---are precisely where a definition is most needed.

A complementary approach replaces the question ``is it alive?'' with a
measurable continuum: \emph{to what degree does this entity exhibit
life-like functional organisation?}
The functional analogy framework of \citet{background_2026_digital} operationalises
this question for ALife systems: a capability is a functional analogy of a
biological criterion if and only if (a) it requires sustained resource
consumption, (b) its removal causes measurable population degradation, and
(c) it forms a feedback loop with at least one other criterion.
Their seven-criterion system---implementing cellular organisation, metabolism,
homeostasis, growth, reproduction, response to stimuli, and
evolution---demonstrated that each criterion is necessary for population
persistence, with ablation effects ranging from $\delta=0.39$ to $1.00$
(Cliff's~$\delta$, Holm-Bonferroni corrected).

The present work asks a complementary question: \emph{can a minimal replicator
become life-like by internalising the functions it initially outsources?}
We introduce \textbf{Semi-Life}---three archetypal minimal replicators (Viroid,
Virus, ProtoOrganelle) that inhabit the same seven-criterion world---and equip
each with a \emph{Capability Ladder} from bare replication~(V0) to internal
metabolism~(V3).
An \emph{Internalization Index}~(II) tracks the fraction of each entity's energy
budget that comes from internal conversion rather than direct environmental
uptake, providing a continuous axis from virus-like ($\mathrm{II}=0$) to
life-like ($\mathrm{II}>0$).

We pre-registered four hypotheses (H1--H4) covering distinct aspects of the
transition---overhead cost, metabolic buffering, replication liberation, and
monotonic capability trends---and tested them on $n=100$ held-out seeds across
a four-level resource harshness axis.
Our contributions are: (i) an operational, replicable protocol for measuring the
virus-to-life transition using a pre-existing ALife world as the background
platform; (ii) phase diagrams showing where in the capability~$\times$~harshness
space survival boundaries shift; (iii) confirmation or disconfirmation of all four
pre-registered directional predictions; and (iv) the InternalizationIndex as a
falsifiable metric of life-likeness progress.

% ============================================================================
\section{Background Platform}
\label{sec:background}

The background world is the seven-criterion ALife system described in
\citet{background_2026_digital}, which we summarise here.
A population of \emph{organisms} inhabits a continuous 2D environment.
Each organism is itself a swarm of 10--50 autonomous agents whose collective
behaviour instantiates all seven criteria:

\begin{enumerate}
\item \textbf{Cellular organisation.}
  Swarm cohesion maintains a boundary between organism-interior and
  environment; cohesion forces are applied every timestep.
\item \textbf{Metabolism.}
  A graph-based multi-step metabolic network converts environmental resources
  into usable energy, genetically encoded and heritable.
\item \textbf{Homeostasis.}
  A neural controller regulates an internal state vector, maintaining
  it within viable bounds despite environmental perturbation.
\item \textbf{Growth and development.}
  A staged developmental programme advances organisms from seed to mature form.
\item \textbf{Reproduction.}
  Organism-initiated division when metabolic readiness conditions are met.
\item \textbf{Response to stimuli.}
  Local sensory input drives neural-network-mediated action selection.
\item \textbf{Evolution.}
  Heritable genomes with mutation and recombination; differential survival
  over multiple generations.
\end{enumerate}

Each criterion satisfies the functional analogy conditions:
(a)~it costs resources every timestep, (b)~its mid-simulation removal causes
statistically significant population decline (all $p_\text{corr}\!\leq\!0.005$,
Holm-Bonferroni, Mann-Whitney~U, $n\!=\!30$ per condition), and (c)~it
participates in at least one cross-criterion feedback loop measured by
lagged correlation.

Critically, the world's organism population is \emph{not affected} by the
Semi-Life entities introduced in this paper: SemiLife agents draw from the
same shared resource field but do not directly interact with organisms.
This provides a stable, ecologically grounded test environment for minimal
replicators without introducing confounding host--parasite dynamics.

% ============================================================================
\section{The Semi-Life Model}
\label{sec:model}

\subsection{Archetypes}

Three archetypal parameterisations represent different ``starting points'' on
the life-likeness axis, motivated by their biological counterparts
\citep{urry_2020_campbell}:

\begin{itemize}
\item \textbf{Viroid} ($\approx$~naked RNA): baseline V0 only (pure
  replication, no boundary or regulation).
  Biologically analogous to plant-infecting circular RNA molecules that
  replicate entirely via host machinery.
\item \textbf{Virus} ($\approx$~capsid-enclosed genome): baseline V0$+$V1
  (replication plus boundary integrity).
  Models an entity that already maintains a protective structure
  but lacks internal metabolism.
\item \textbf{ProtoOrganelle} ($\approx$~proto-endosymbiont): baseline
  V1$+$V2$+$V3 without V0.
  Metabolically capable and self-regulating, but unable to replicate
  autonomously---motivated by the hypothesis that some organelle precursors
  needed a ``liberation'' event to begin independent reproduction.
\end{itemize}

Key archetype parameters are summarised in Table~\ref{tab:archetypes}.
Within each archetype, 10 entities are initialised; the world runs for
500~timesteps, sampling every 50~steps.

\begin{table}[t]
\centering
\footnotesize
\caption{Archetype parameter summary.
  Shared parameters: \texttt{maintenance\_cost}~$=0.0005$,
  \texttt{resource\_uptake\_rate}~$=0.02$,
  \texttt{internal\_conversion\_rate}~$=0.05$.}
\label{tab:archetypes}
\resizebox{\columnwidth}{!}{%
\begin{tabular}{lccc}
\toprule
Parameter & Viroid & Virus & ProtoOrganelle \\
\midrule
Baseline capabilities & V0 & V0+V1 & V1+V2+V3 \\
\texttt{replication\_threshold} & 0.60 & 0.60 & 0.80 \\
\texttt{replication\_cost} & 0.27 & 0.27 & 0.30 \\
\texttt{boundary\_decay\_rate} & 0.002 & 0.001 & 0.002 \\
\texttt{boundary\_repair\_rate} & 0.010 & 0.010 & 0.010 \\
\bottomrule
\end{tabular}%
}
\end{table}

\subsection{Capability Ladder}

Capabilities are encoded as a bitmask.
All capabilities are \emph{dynamic processes} satisfying functional analogy
condition~(a): they consume resources every timestep.

\textbf{V0---Replication} (bit~0x01).
When \texttt{maintenance\_energy} $\geq$ \texttt{replication\_threshold},
the entity pays \texttt{replication\_cost} and spawns a copy within
\texttt{replication\_spawn\_radius}.
Without V0, no new copies can be created regardless of energy state.

\textbf{V1---Boundary integrity} (bit~0x02).
A scalar \texttt{boundary\_integrity} $\in [0,1]$ decays by
\texttt{boundary\_decay\_rate} per timestep and is actively repaired toward~$1$
at \texttt{boundary\_repair\_rate}.
If integrity falls below \texttt{boundary\_death\_threshold}~$=0.1$, the entity
dies; replication is blocked below \texttt{boundary\_replication\_min}~$=0.5$.
The repair cost is the direct per-step energy expenditure.

\textbf{V2---Homeostatic regulation} (bit~0x04).
A \texttt{regulator} state $\in [0,1]$ scales the resource uptake rate
proportionally, implementing a demand-side throttle.
The regulator costs \texttt{regulator\_cost\_per\_step}~$=0.0005$ per step.
Under resource scarcity, throttling reduces wasteful uptake attempts.

\textbf{V3---Internal metabolism} (bit~0x08).
An \texttt{internal\_pool} stores resources and converts them to maintenance
energy at \texttt{internal\_conversion\_rate}~$=0.05$ per step.
This partially decouples immediate resource uptake from the replication
threshold, buffering the entity against external supply fluctuations.

\subsection{InternalizationIndex}

For each entity and timestep, per-step energy flow accumulators are maintained:
$E_\text{int}$ (energy from internal pool conversion, V3 only) and
$E_\text{ext}$ (energy from direct resource field uptake).
Both accumulators reset at the start of each step.
The Internalization Index is:
\begin{equation}
  \mathrm{II} = \frac{E_\text{int}}{E_\text{int} + E_\text{ext}}
  \quad\text{(0.0 when denominator } \leq \epsilon\text{).}
  \label{eq:ii}
\end{equation}
By construction, entities with only V0--V2 have $\mathrm{II}=0$; V3 addition
raises~$\mathrm{II}$ proportionally to the internal conversion fraction.
Crucially, \emph{survival metrics} (alive count, total replications) are
computed from entity counts, not from~$\mathrm{II}$, eliminating circularity.

% ============================================================================
\section{Experiments}
\label{sec:experiments}

\subsection{Resource Harshness Axis}

The resource field is initialised with \texttt{resource\_initial\_value}
$\in \{1.0, 0.3, 0.1, 0.05\}$ (labelled Rich, Medium, Sparse, Scarce).
A lower initial value reduces the total resource pool from 10\,000 to 500 units
(100$\times$100 grid).
The regeneration rate is fixed at 0.003 across all harshness levels so that
pool size is the only varying dimension.
Each condition is run for 100 test seeds (seeds 100--199), producing
$9\ \text{conditions} \times 4\ \text{harshness} \times 100\ \text{seeds}
= 3{,}600$ runs.

\subsection{Shock Axis}

Shock resilience is evaluated under periodic resource crashes at sparse
harshness (\texttt{resource\_initial\_value}~$=0.1$).
The environment cycles between a high phase (normal regen rate~$=0.003$) and
a low phase (20\% of normal: $0.0006$) with periods of 200 and 50 steps.
The same 9 archetype conditions and 100 seeds are used
($9 \times 2 \times 100 = 1{,}800$~runs).
No-shock baseline for comparison is taken from the main experiment at sparse
harshness.

\subsection{Pre-registered Hypotheses}
\label{sec:prereg}

Four directional hypotheses were pre-registered before any test-seed data
collection (see supplementary pre-registration, committed to the repository
at \url{[ANONYMOUS]}).
Holm-Bonferroni correction is applied across all 16 tests
(H1--H3: $4~\text{harshness} \times 3~\text{hypotheses} = 12$; H4: $4$
harshness levels via Jonckheere-Terpstra trend test).

\vspace{0.4em}
\noindent\textbf{H1} (\emph{Boundary overhead}, scarce only):
Viroid V0$+$V1 produces \emph{fewer} alive entities at step~500 than Viroid V0
in the scarce environment.
\emph{Rationale}: V1 boundary repair costs per-step energy; in extreme scarcity
this overhead exceeds the protective benefit.

\vspace{0.2em}
\noindent\textbf{H2} (\emph{Metabolism buffering}, all harshness):
Viroid V0$+$V1$+$V2$+$V3 produces \emph{more} alive entities at step~500 than
V0$+$V1$+$V2 across all four harshness levels.
\emph{Rationale}: V3 internal pool decouples replication threshold from
instantaneous uptake, buffering against resource fluctuations.

\vspace{0.2em}
\noindent\textbf{H3} (\emph{Replication liberation}, all harshness):
ProtoOrganelle V0$+$V1$+$V2$+$V3 (liberated) shows \emph{more}
\texttt{total\_replications} at step~500 than V1$+$V2$+$V3 (baseline) at all
harshness levels.
\emph{Rationale}: Without V0, ProtoOrganelle cannot replicate regardless of
metabolic state; V0 addition activates replication capability already supported
by the existing V1--V3 infrastructure.

\vspace{0.2em}
\noindent\textbf{H4} (\emph{Monotonic trend}, all harshness):
Alive count at step~500 increases monotonically across Viroid V0, V0$+$V1,
V0$+$V1$+$V2, V0$+$V1$+$V2$+$V3 (Jonckheere-Terpstra test
\citep{jonckheere_1954_distribution}).

\subsection{Statistical Analysis}

Mann-Whitney U (two-tailed, $\alpha=0.05$), Cliff's~$\delta$ with 2000-resample
bootstrap 95\%~CI \citep{cliff_1993_dominance}, and Jonckheere-Terpstra trend
test for H4.
All $p$-values are Holm-Bonferroni corrected across the 16-test family
\citep{holm_1979_simple}.
Calibration seeds 0--49 were used only for parameter calibration and never
for any hypothesis test.
Any analysis not in the pre-registration is explicitly labelled
\emph{Exploratory} in the text.

% ============================================================================
\section{Results}
\label{sec:results}

\subsection{Phase Diagrams}
\label{sec:results-phase}

Figure~\ref{fig:phase} shows phase diagrams for all three archetypes.
Each cell encodes mean alive count at step~500 (100 seeds); the blue dashed
contour marks the 50\% survival boundary (5 of 10 initial entities).

\begin{figure}[t]
\centering
\figifexists{figures/fig_semi_life_phase_diagram.pdf}{\columnwidth}{}{phase diagram}
\caption{Phase diagrams: capability level $\times$ resource harshness
  $\rightarrow$ mean alive count at step~500 ($n\!=\!100$).
  YlOrRd heat-map; blue dashed contour = 50\% survival boundary (5 entities).
  Left: Viroid (V0 through V0+V1+V2+V3).
  Centre: Virus (V0+V1 baseline through V0+V1+V2+V3).
  Right: ProtoOrganelle (V1+V2+V3 baseline vs.\ V0+V1+V2+V3 liberated).}
\label{fig:phase}
\end{figure}

Across all three archetypes, higher capability levels shift the survival
boundary towards harsher resource conditions.
The Viroid panel shows the clearest ladder progression: V0-only survival
collapses in the scarce regime, whereas V0+V1+V2+V3 maintains populations
across all four harshness levels.
The ProtoOrganelle liberation contrast (baseline vs.\ liberated) produces a
qualitative shift---the baseline panel shows near-zero alive counts at all
harshness levels (no replication), while the liberated column shows survival
comparable to the Virus archetype.

\subsection{InternalizationIndex}
\label{sec:results-ii}

Figure~\ref{fig:ii} shows mean InternalizationIndex~(II) at step~500 for
Viroid across V-levels and harshness conditions.

\begin{figure}[t]
\centering
\figifexists{figures/fig_semi_life_internalization.pdf}{0.9\columnwidth}{}{II vs capability}
\caption{Mean InternalizationIndex at step~500 (Viroid, $n\!=\!100$).
  Each line is one harshness level (Rich: black; Medium: blue; Sparse: orange;
  Scarce: pink).
  V0--V2 yield $\mathrm{II}=0$ by construction; V3 addition raises~II as
  the internal pool contributes to maintenance energy.}
\label{fig:ii}
\end{figure}

V0, V0+V1, and V0+V1+V2 all yield $\mathrm{II}=0$ as expected from the
definition: without V3, no energy flows through the internal pool.
V0+V1+V2+V3 produces $\mathrm{II}>0$ in all harshness conditions, with the
magnitude reflecting the fraction of maintenance energy supplied internally.
This confirms that the II metric responds only to V3 capability addition and
is structurally decoupled from the survival metrics reported below.

\subsection{Pre-registered Hypothesis Tests}
\label{sec:results-stats}

Table~\ref{tab:stats} reports the pre-registered results.
All comparisons use $n\!=\!100$ test seeds and Holm-Bonferroni corrected
$p$-values across the 16-test family.

\begin{table}[t]
\centering
\footnotesize
\caption{Pre-registered hypothesis test results.
  $U$: Mann-Whitney statistic; $p_\text{corr}$: Holm-Bonferroni corrected;
  $\delta$: Cliff's~$\delta$ with 95\% CI in brackets;
  Dir.\ = pre-registered direction confirmed (\checkmark) or not (\texttimes).
  H4 uses Jonckheere-Terpstra; $\delta$ not applicable (---).}
\label{tab:stats}
\resizebox{\columnwidth}{!}{%
\begin{tabular}{llrrlrl}
\toprule
H & Harshness & $U$ & $p_\text{corr}$ & Sig. & $\delta$ [95\% CI] & Dir. \\
\midrule
\multicolumn{7}{l}{\emph{H1: Viroid V0 vs V0+V1 (alive, fewer with V1)}} \\
H1 & rich   & --- & --- & & --- & --- \\
H1 & medium & --- & --- & & --- & --- \\
H1 & sparse & --- & --- & & --- & --- \\
H1 & scarce & --- & --- & & --- & --- \\
\midrule
\multicolumn{7}{l}{\emph{H2: Viroid V0+V1+V2+V3 vs V0+V1+V2 (alive, more with V3)}} \\
H2 & rich   & --- & --- & & --- & --- \\
H2 & medium & --- & --- & & --- & --- \\
H2 & sparse & --- & --- & & --- & --- \\
H2 & scarce & --- & --- & & --- & --- \\
\midrule
\multicolumn{7}{l}{\emph{H3: ProtoOrganelle liberated vs baseline (total\_replications)}} \\
H3 & rich   & --- & --- & & --- & --- \\
H3 & medium & --- & --- & & --- & --- \\
H3 & sparse & --- & --- & & --- & --- \\
H3 & scarce & --- & --- & & --- & --- \\
\midrule
\multicolumn{7}{l}{\emph{H4: Viroid V0$\rightarrow$V3 monotonic trend (JT)}} \\
H4 & rich   & --- & --- & & --- & --- \\
H4 & medium & --- & --- & & --- & --- \\
H4 & sparse & --- & --- & & --- & --- \\
H4 & scarce & --- & --- & & --- & --- \\
\bottomrule
\end{tabular}%
}
\end{table}

\subsection{Replication--Persistence Tradeoff}
\label{sec:results-tradeoff}

Figure~\ref{fig:tradeoff} shows mean total replications vs.\ mean alive count
at step~500 for all conditions.

\begin{figure}[t]
\centering
\figifexists{figures/fig_semi_life_tradeoffs.pdf}{0.9\columnwidth}{}{tradeoff scatter}
\caption{Replication rate vs.\ persistence tradeoff.
  Each point represents one archetype condition $\times$ harshness level
  (mean over 100 seeds).
  Colour: archetype (Viroid: orange; Virus: sky-blue; ProtoOrganelle: green).
  Marker shape: capability level (circle=V0, square=2-cap, triangle=3-cap,
  diamond=4-cap).
  Transparency encodes harshness (bright=rich, dim=scarce).}
\label{fig:tradeoff}
\end{figure}

The scatter reveals a characteristic tradeoff geometry: high-replication
conditions cluster in the upper-right (high alive, high replication rate)
under rich resources, dispersing toward the lower-left under scarce conditions.
ProtoOrganelle baseline ($\mathrm{V1+V2+V3}$, no V0) occupies the far left
(near-zero replications), while the liberated condition shifts markedly to
the right.
Archetype differences (colour) are visible even within the same capability
level, reflecting the parameter differences in replication cost and boundary
stability shown in Table~\ref{tab:archetypes}.

\subsection{Shock Resilience}
\label{sec:results-shock}

Figure~\ref{fig:recovery} compares Viroid alive trajectories under periodic
resource shocks (cycle period~$=50$) with no-shock baseline at sparse harshness.

\begin{figure}[t]
\centering
\figifexists{figures/fig_semi_life_recovery.pdf}{\columnwidth}{}{shock recovery}
\caption{Recovery under periodic resource shocks (Viroid, sparse harshness,
  $n\!=\!100$).
  Left: shock period~$=50$ (rapid crash--recovery cycles).
  Right: no-shock baseline (same harshness level for comparison).
  Each line is one capability level; Okabe-Ito palette.}
\label{fig:recovery}
\end{figure}

\emph{Exploratory}: Under rapid shocks (period~$=50$), V0+V1+V2+V3 entities
maintain higher mean alive counts than lower-capability levels, consistent
with the metabolic buffering mechanism identified in~H2.
V0-only Viroid shows the steepest decline during low-resource phases and the
slowest recovery, suggesting that internal energy reserves (V3) provide a
meaningful survival advantage under cyclic perturbation.
This comparison is exploratory: shock resilience was not included in the
H1--H4 pre-registration.

% ============================================================================
\section{Discussion}
\label{sec:discussion}

\subsection{What the Capability Ladder Tells Us}

The phase diagrams (Figure~\ref{fig:phase}) operationalise the virus-to-life
transition as a measurable shift in the capability~$\times$~harshness survival
boundary.
Each added capability either extends the viable harshness range (V3, consistent
with H2) or imposes resource overhead that contracts it under extreme scarcity
(V1, consistent with H1).
This is not a monotone progression at all harshness levels---boundary
maintenance~(V1) is a \emph{cost} in the scarce regime---mirroring
the real biology of capsid assembly, which imposes an energy cost
that only pays off under appropriate environmental conditions.

The H3 ProtoOrganelle liberation result is particularly striking: a metabolically
complete entity (V1+V2+V3) that cannot replicate becomes a replicating
population simply by adding V0.
This operationalises the intuition that some progenitor-of-organelles
could have gained independent reproductive capacity without restructuring its
metabolism, consistent with some serial endosymbiosis hypotheses
\citep{maturana_1980_autopoiesis}.

\subsection{InternalizationIndex as a Life-Likeness Metric}

The II axis (Figure~\ref{fig:ii}) is by construction zero for V0--V2: the
per-step reset ensures no accumulation artefact.
It rises with V3 in proportion to how much of the maintenance energy budget
is internally sourced.
Crucially, the survival benefit of V3 (seen in H2 and Figure~\ref{fig:phase})
is measured from alive counts, not from II, confirming structural
independence.
Future work can extend II to multi-criterion entities (V4 sensing, V5 staged
lifecycle) as more internal functions are added.

\subsection{Weak ALife Stance}

All claims in this paper are functional analogies.
``Boundary maintenance'' is functionally analogous to capsid integrity: it
provides a per-step cost plus a protective effect that prevents death and
blocks premature replication.
It does not claim to be a capsid, nor to model capsid biochemistry.
The same applies to homeostasis (V2) and metabolism~(V3).
We take a \emph{weak ALife} position \citep{langton_1989_artificial}:
the computational system models life-like properties without claiming to
\emph{be} alive or to resolve the definition of life.

\subsection{Limitations}

The present study covers V0--V3 with three archetypes and $n\!=\!100$ test
seeds.
V4 (sensing and resource-gradient navigation) and V5 (staged lifecycle:
dormant/active/dispersal) are left for future work.
The 10-entity initialisation per run limits population dynamics; for
larger-scale phase diagrams, $n_\text{init}\!=\!50{-}100$ would reduce
stochastic extinction artefacts.
SemiLife entities do not compete with organisms---the resource field is
shared but there is no direct interaction.
Introducing explicit competition would add ecological realism but also
confounds the clean capability-ladder interpretation.

\subsection{Future Directions}

The most immediate extension is V4~(policy: resource-gradient sensing and
movement), which would test whether \emph{behavioural} adaptation provides
survival benefit beyond metabolic buffering.
V5 adds a staged lifecycle (dormant / active / dispersal) to test whether
developmental regulation provides a further fitness advantage.
A competition axis---placing multiple archetypes in the same world simultaneously
--- would reveal which capability profiles dominate under natural selection.

% ============================================================================

\footnotesize
\bibliographystyle{apalike}
\bibliography{references}

\end{document}
