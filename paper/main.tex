\documentclass[letterpaper]{article}
\usepackage{natbib,alifeconf}
\usepackage{graphicx,amsmath,amssymb,booktabs,xcolor,multirow}
\usepackage{url,hyperref}

\newcommand\blfootnote[1]{%
  \begingroup
  \renewcommand\thefootnote{}\footnote{#1}%
  \addtocounter{footnote}{-1}%
  \endgroup
}

\title{Digital Life: Implementing Seven Biological Criteria\\Through Functional Analogy and Criterion-Ablation}
\author{Anonymous}

\begin{document}
\maketitle

% ============================================================================
\begin{abstract}
We present a testable integration implementing all seven textbook biological
criteria for life---cellular organization, metabolism, homeostasis, growth,
reproduction, response to stimuli, and evolution---as functionally
interdependent computational processes within a single artificial life system.
Existing systems implement at most a subset of these criteria, often as
independent modules or static proxies that can be removed without measurable
system degradation.
Our hybrid swarm-organism architecture implements each criterion as a dynamic
process satisfying three conditions---sustained resource consumption,
measurable degradation upon removal, and feedback coupling with other
criteria---which we term \emph{functional analogy}.
A criterion-ablation experiment ($n$=30 per condition, seeds held out from
calibration) demonstrates that disabling any single criterion causes
statistically significant population decline (Mann-Whitney $U$,
Holm-Bonferroni corrected, all $p \leq 0.016$), with Cliff's $\delta$
ranging from 0.32 to 1.00.
Pairwise ablations reveal sub-additive interactions consistent with
shared failure pathways, confirming that criteria damage overlapping
subsystems rather than operating independently.
A proxy control comparison shows that metabolism implementations of
differing complexity produce qualitatively distinct population dynamics,
ruling out tautological criterion definitions.
The strongest single-ablation effects arise from disabling reproduction
($\Delta$=--89.3\%), response to stimuli ($\Delta$=--88.4\%), and metabolism
($\Delta$=--84.3\%), confirming that these criteria function as necessary,
interdependent components of organismal viability rather than decorative
labels.
\end{abstract}

Submission type: \textbf{Full Paper}\\
Data/Code available at: \url{https://anonymous.4open.science/}
\blfootnote{\textcopyright\ 2026 Authors. Published under a Creative Commons Attribution 4.0 International (CC BY 4.0) license.}

% ============================================================================
\section{Introduction}

What distinguishes a living system from a merely complex one?
Biology textbooks identify seven criteria---cellular organization, metabolism,
homeostasis, growth and development, reproduction, response to stimuli, and
evolution \citep{urry_2020_campbell}---but artificial life (ALife) research has
struggled to integrate all seven into a single computational system.
Most existing platforms implement a subset: evolutionary dynamics without
metabolism \citep{ofria_2004_avida}, pattern formation without reproduction
\citep{chan_2019_lenia}, or boundary self-organization without evolution
\citep{plantec_2023_flow}.
Where criteria are nominally present, they often function as \emph{simplified
proxies}---static parameters or independent modules whose removal has no
measurable effect on system behavior.

We argue that a meaningful computational model of life requires more than
feature checklists.
Each criterion must function as a \emph{functional analogy} of its biological
counterpart, satisfying three conditions:
\begin{enumerate}
  \item \textbf{Dynamic process}: the criterion requires sustained resource
    consumption at every timestep, not a static lookup.
  \item \textbf{Measurable degradation}: ablating the criterion causes
    statistically significant decline in organism viability.
  \item \textbf{Feedback coupling}: the criterion forms at least one feedback
    loop with another criterion, precluding independent-module implementations.
\end{enumerate}
This definition operationalizes the intuition behind autopoiesis
\citep{maturana_1980_autopoiesis}, the chemoton model's integration
of metabolic, boundary, and information subsystems
\citep{ganti_2003_principles}, and minimal-life frameworks
\citep{ruizmirazo_2004_universal} in a form amenable to experimental
falsification.

We adopt a \emph{weak ALife} stance: our system is a functional model of life,
not a claim that the organisms are alive.
The contribution is methodological---demonstrating that all seven criteria
\emph{can} be integrated as interdependent processes and that their necessity
can be rigorously tested.

This paper contributes: (1)~a hybrid swarm-organism architecture integrating
all seven criteria as functionally interdependent processes;
(2)~an operational definition of \emph{functional analogy} with three
falsifiable conditions; (3)~a \emph{criterion-ablation} methodology
demonstrating each criterion's functional necessity with multiple-comparison
correction; and (4)~pairwise ablation and proxy control experiments
characterizing criterion interactions and ruling out tautological definitions.
The key contribution is not a new digital organism per se, but a falsifiable
experimental framework for testing the functional necessity and interaction
of life criteria in any ALife system.

% ============================================================================
\section{Related Work}

Existing ALife systems each excel along different axes of biological fidelity.
Tierra \citep{ray_1991_approach} and Avida \citep{ofria_2004_avida} achieve
strong evolutionary dynamics but lack spatial bodies and metabolism.
Polyworld \citep{yaeger_1994_computational} adds NN-driven behavior and
single-resource energy budgets.
Lenia \citep{chan_2019_lenia} and Flow-Lenia \citep{plantec_2023_flow}
demonstrate emergent spatial organization in continuous cellular automata,
with Flow-Lenia adding mass conservation.
ALIEN \citep{heinemann_2008_alien} provides a GPU-accelerated particle
simulator achieving multi-process interaction on several criteria, though
its metabolism uses typed particle interactions rather than a multi-step
metabolic network (Level~3 on our rubric).
Coralai \citep{barbieux_2024_coralai} combines multi-agent neural cellular
automata with energy dynamics but lacks feedback coupling between criteria.
No existing system combines multi-step metabolism with active homeostatic
regulation.

Theoretically, our framework operationalizes autopoiesis
\citep{maturana_1980_autopoiesis, mcmullin_2004_thirty},
extends the chemoton's three-subsystem integration
\citep{ganti_2003_principles}, and enriches NASA's two-criterion
definition \citep{joyce_1994_foreword} following
\citet{ruizmirazo_2004_universal}.
Open-ended evolution metrics \citep{bedau_2000_open,
taylor_2016_openended} complement our criterion-ablation approach.

Table~\ref{tab:comparison} summarizes how existing systems score on each
criterion using a five-level rubric.

\begin{table*}[t]
\centering
\caption{Literature comparison: seven biological criteria scored on a
five-level rubric (1=no feature, 2=static parameter, 3=dynamic single process,
4=multi-process interaction, 5=self-maintaining/emergent).
Bold indicates scores $\geq$4.}
\label{tab:comparison}
\small
\begin{tabular}{l@{\hskip 6pt}c@{\hskip 6pt}c@{\hskip 6pt}c@{\hskip 6pt}c@{\hskip 6pt}c@{\hskip 6pt}c@{\hskip 6pt}c@{\hskip 6pt}c}
\toprule
System & Cell.Org & Metab & Homeo & Growth & Reprod & Response & Evol & Total \\
\midrule
% Scores verified via literature review; justifications in paper/literature_scores.md
Polyworld  & 2 & 3 & 1 & 1 & 3 & \textbf{4} & \textbf{4} & 18 \\
Avida      & 2 & 3 & 1 & 2 & \textbf{4} & 3 & \textbf{5} & 20 \\
Lenia      & 3 & 1 & 2 & 2 & 2 & 3 & 2 & 15 \\
ALIEN      & \textbf{4} & 3 & 2 & 3 & \textbf{4} & \textbf{4} & \textbf{4} & 24 \\
Flow-Lenia & 3 & 3 & 3 & 3 & 3 & 3 & 3 & 21 \\
Coralai    & 3 & 3 & 2 & 3 & 3 & 3 & 3 & 20 \\
\midrule
\textbf{Ours}$^\dagger$ & \textbf{4} & \textbf{4} & \textbf{4} & 3 & \textbf{4} & \textbf{4} & 3 & \textbf{26} \\
\bottomrule
\multicolumn{9}{l}{\footnotesize $^\dagger$Self-assessment; scores may differ under external evaluation.}\\
\multicolumn{9}{l}{\footnotesize Growth (3) and Evolution (3) represent minimum viable implementations}\\
\multicolumn{9}{l}{\footnotesize satisfying the functional-analogy conditions.}\\
\multicolumn{9}{l}{\footnotesize Scores based on published system descriptions; per-criterion justifications}\\
\multicolumn{9}{l}{\footnotesize available as supplementary material.}
\end{tabular}
\end{table*}

Our system reaches $\geq$4 on five of seven criteria, placing it among the
most broadly integrated ALife systems surveyed.
We note that these scores are self-assessed; independent evaluation may adjust
individual ratings.

% ============================================================================
\section{System Design}

\subsection{Architecture Overview}

The system implements a hybrid two-layer architecture (Figure~\ref{fig:arch}).
The outer layer is a continuous toroidal 2D environment (100$\times$100 world
units) containing a diffusing resource field.
The inner layer consists of 10--50 \emph{organisms}, each composed of 10--50
\emph{swarm agents} that collectively maintain the organism's spatial boundary.

\begin{figure*}[t]
\centering
\includegraphics[width=\textwidth]{figures/fig_architecture.pdf}
\caption{Two-layer architecture. Each organism comprises swarm agents maintaining
a spatial boundary, a neural-network controller, a graph-based metabolic network,
and a variable-length genome encoding all seven criteria. Organisms inhabit a
continuous toroidal environment with a diffusing resource field.}
\label{fig:arch}
\end{figure*}

Each organism maintains the following runtime state: boundary integrity
($b \in [0,1]$), metabolic state (energy $e$, resource $r$, waste $w$),
internal state vector for homeostatic regulation, a neural-network controller,
a genetically encoded metabolic network, age, generation counter, and maturity
level.

\subsection{Seven Criteria Implementation}

Table~\ref{tab:criteria} maps each biological criterion to its computational
implementation, ablation toggle, and feedback partners.

\begin{table*}[t]
\centering
\caption{Mechanism specification: state variables, ablation operators, coupling
pathways, and failure modes for each criterion. All criteria satisfy the three
functional-analogy conditions (dynamic process, measurable degradation,
feedback coupling).}
\label{tab:criteria}
\footnotesize
\begin{tabular}{@{}p{1.5cm}p{2.0cm}p{2.6cm}p{4.0cm}p{2.2cm}@{}}
\toprule
Criterion & State Vars & Ablation & Coupling & Failure \\
\midrule
Cell.\ Org. & $b \in [0,1]$ & Skip repair; decay only & $e \to$ repair rate $\to b$ & $b{<}0.1 \to$ death \\
Metab. & $e,r,w$; graph $\mathbf{p}$ & Freeze $(e,r,w)$ & throughput $\to e \to$ bdry repair & Energy depletion \\
Homeo. & $\mathbf{s} \in \mathbb{R}^3$ & Skip NN delta; decay & $s_0 \to$ repair efficacy & State drift \\
Growth & $m \in [0,1]$ & Freeze $m{=}0$ & maturity $\to$ reprod.\ gate & No reproduction \\
Reprod. & Pop.\ event & Skip division check & energy cost $\to$ parent $e$ & No replacement \\
Response & $\mathbf{v}$ & Skip velocity delta & movement $\to$ resource $\to r$ & Starvation \\
Evolution & genome $\mathbf{g}$ & Copy w/o mutation & $\mathbf{g}$ variation $\to$ all & No adaptation \\
\bottomrule
\end{tabular}
\end{table*}

\paragraph{Cellular organization.}
Swarm agents collectively define an organism's spatial extent.
Boundary integrity $b$ decays each step at a base rate modulated by energy
deficit and waste pressure:
$\Delta b_{\text{decay}} = -r_b \cdot (1 + s_e \cdot (1 - e) + s_w \cdot w)$,
where $r_b = 0.02$ is the base decay rate, $s_e = 0.5$ and $s_w = 0.3$ are
scaling factors.
Repair occurs proportionally to available energy:
$\Delta b_{\text{repair}} = r_r \cdot e \cdot (1 - s_p \cdot w)$,
with repair rate $r_r = 0.15$ and waste penalty $s_p = 0.4$.
When $b$ falls below a collapse threshold ($b < 0.1$), the organism dies.

\paragraph{Metabolism.}
Each organism possesses a genetically encoded graph-based metabolic network
with 2--4 catalytic nodes and directed edges.
The genome segment (16 floats) is decoded via sigmoid mapping:
node count $= \text{round}(\sigma(g_0) \cdot 2 + 2)$,
catalytic efficiency $= \sigma(g_{2+i}) \cdot 0.9 + 0.1 \in [0.1, 1.0]$,
edge existence determined by $|g_j| > 0.3$, and
conversion efficiency $= \sigma(g_{13}) \cdot 0.7 + 0.3 \in [0.3, 1.0]$.
External resources enter at a designated entry node, flow through the graph
with per-edge transfer efficiency in $[0.7, 1.0]$, and exit as energy.
Waste accumulates as a byproduct proportional to throughput.

\paragraph{Homeostasis.}
A feedforward neural network (8 inputs $\rightarrow$ 16 hidden with tanh
$\rightarrow$ 4 outputs with tanh; 212 weights) processes sensory inputs
(normalized position $x,y$; velocity $v_x, v_y$; internal state
$s_0, s_1, s_2$; neighbor density) and produces velocity
adjustments and internal-state deltas.
The internal state vector enables adaptive regulation: organisms that maintain
internal variables within viable ranges survive longer.

\paragraph{Growth and development.}
Organisms begin as minimal seeds (maturity $m = 0$) and develop toward full
capacity ($m = 1$) over time.
Maturation gates metabolic throughput and reproductive readiness, ensuring
organisms must develop before they can reproduce.

\paragraph{Reproduction.}
When energy exceeds $e_{\min} = 0.7$ and boundary integrity exceeds
$b_{\min} = 0.5$, an organism may divide.
The parent pays an energy cost ($c_r = 0.3$), and the offspring inherits a
(possibly mutated) copy of the genome, starting as a seed.
Child agents spawn within a radius of the parent's center of mass.

\paragraph{Response to stimuli.}
The neural-network controller processes a local sensory field each timestep,
producing velocity deltas that govern agent movement.
Disabling response freezes agents' velocity adjustments, preventing adaptive
resource seeking.

\paragraph{Evolution.}
During reproduction, offspring genomes undergo point mutations
(rate $= 0.01$ per gene, scale $= 0.1$), reset mutations
(rate $= 0.001$), and scale mutations (rate $= 0.005$,
factor $\in [0.8, 1.2]$).
All gene values are clamped to $[-5, 5]$.
This produces heritable variation subject to differential survival.

\subsection{Genome Encoding}

The genome is a variable-length vector of 256 floats organized into seven
segments: neural-network weights (212), metabolic network (16), homeostasis
parameters (8), developmental program (8), reproduction parameters (4),
sensory parameters (4), and evolution parameters (4).
All criteria are encoded from initialization; segments are activated as
features are enabled.

% ============================================================================
\section{Criterion-Ablation Experiment}

This experiment tests whether each of the seven criteria is functionally
necessary for organism viability, as predicted by the functional-analogy
framework.

\subsection{Protocol}

The system provides seven boolean ablation toggles, one per criterion
(e.g., \texttt{enable\_metabolism = false}).
For each of the seven criteria, we disable that criterion while keeping all
others active, and compare the resulting population dynamics against the
fully enabled baseline (``normal'' condition).
This yields eight conditions: one normal baseline and seven single-criterion
ablations.

\subsection{Outcome Measures}

An \emph{organism} is a persistent runtime entity with a unique ID;
offspring receive a new ID at division; no merge or split occurs.
An organism dies when boundary $b < 0.1$, energy $e \leq 0.0$, or
age $> 20{,}000$ steps.
The \textbf{primary dependent variable} is $N_T$, the alive organism count
at the final step $T=2000$; each seed maps to one scalar.
Secondary DVs include the area under the alive-count curve (AUC,
trapezoidal rule) and median organism lifespan.
We additionally report short-horizon viability at $T=500$ to separate
individual survival from population replacement effects.

\subsection{Data Separation}

To prevent overfitting of thresholds, we separate data into:
\begin{itemize}
  \item \textbf{Calibration set}: Seeds 0--99, used during development for
    parameter tuning and threshold selection.
  \item \textbf{Test set}: Seeds 100--129 ($n$=30), held out until final
    evaluation. All reported results use this set exclusively.
\end{itemize}

Calibration confirmed that both metabolism engines produce viable populations
(Toy: $\bar{x}$=328.1, Graph: $\bar{x}$=291.8 alive at step 2000).
All final experiments use the Graph metabolism engine.

\subsection{Simulation Parameters}

Each simulation runs for 2000 timesteps with population sampled every 50
steps.
The environment is a 100$\times$100 toroidal grid with 30 initial organisms,
each comprising 25 swarm agents.
Resources regenerate at 0.01 per cell per step with no diffusion and
toroidal wrap boundary conditions.
No early stopping is applied.
Ablation toggles do not alter timestep duration or event ordering;
disabled processes are simply skipped.
Each seed produces a deterministic, reproducible outcome within a given
condition, though RNG call sequences differ across conditions when
ablated processes (e.g., reproduction) skip conditional RNG draws.

\subsection{Statistical Design}

For each ablation condition, we test the one-sided hypothesis:
\[
H_1: \text{alive\_count}_{\text{normal}} > \text{alive\_count}_{\text{ablated}}
\]
using the Mann-Whitney $U$ test \citep{mann_1947_test}, appropriate for
non-normal count data.
We apply Holm-Bonferroni correction \citep{holm_1979_simple} for seven
simultaneous comparisons at $\alpha = 0.05$.
Effect sizes are reported as Cliff's $\delta$ \citep{cliff_1993_dominance}
for the primary ablation comparisons, with Cohen's $d$ additionally reported
for evolution-strengthening experiments to facilitate parametric comparison,
with percentile bootstrap 95\% confidence intervals (2000 resamples,
seed-fixed RNG), appropriate for non-normal distributions.

% ============================================================================
\section{Results}

All seven criterion ablations produce statistically significant population
decline compared to the normal baseline (Table~\ref{tab:ablation}).

\begin{table}[t]
\centering
\caption{Criterion-ablation results ($n$=30 per condition).
Normal baseline mean: 293.1 (median: 294.5, IQR: 281.5--309.5).
$\delta$=Cliff's delta [bootstrap 95\% CI].
Seven one-sided tests, Holm-Bonferroni corrected at $\alpha=0.05$.
$^{***}p<0.001$, $^{*}p<0.05$.}
\label{tab:ablation}
\small
\begin{tabular}{@{}lr@{\hskip 4pt}r@{\hskip 4pt}l@{\hskip 4pt}r@{\hskip 4pt}r@{}}
\toprule
Condition & Mean & $\Delta$\% & $\delta$ [95\% CI] & $p_{\text{corr}}$ & Sig. \\
\midrule
No Reprod.  & 31.3 & $-$89.3 & 1.00 [1.00, 1.00] & $<$.001 & $^{***}$ \\
No Response & 34.1 & $-$88.4 & 1.00 [1.00, 1.00] & $<$.001 & $^{***}$ \\
No Metab.   & 46.0 & $-$84.3 & 1.00 [1.00, 1.00] & $<$.001 & $^{***}$ \\
No Homeo.   & 68.3 & $-$76.7 & 1.00 [1.00, 1.00] & $<$.001 & $^{***}$ \\
No Boundary & 121.8 & $-$58.5 & 1.00 [1.00, 1.00] & $<$.001 & $^{***}$ \\
No Growth   & 185.6 & $-$36.7 & 1.00 [1.00, 1.00] & $<$.001 & $^{***}$ \\
No Evol.    & 278.3 & $-$5.1  & 0.32 [0.02, 0.59] & .016 & $^{*}$ \\
\bottomrule
\end{tabular}
\end{table}

Three ablations cause near-total population collapse ($>$84\% decline):
reproduction, response to stimuli, and metabolism.
These criteria form the core viability loop---without energy production,
adaptive movement, or population renewal, organisms cannot sustain themselves.

Figure~\ref{fig:timeseries} shows population trajectories across all
conditions.
The normal condition (black) stabilizes around 293 organisms by step~1900.
Metabolic ablation (orange) causes rapid collapse within the first 200 steps,
as organisms cannot produce energy to maintain boundaries.
Reproduction ablation (blue) produces a slower but equally terminal decline,
as the initial population ages and dies without replacement.
Evolution ablation (purple) shows the weakest effect (Cliff's $\delta$=0.32),
with populations remaining viable but slightly smaller than normal---consistent
with evolution operating as an optimization process rather than a survival
necessity at these timescales.
AUC of the alive-count curve corroborates $N_T$ rankings across all
conditions (normal AUC=435{,}376 vs.\ 74{,}545--412{,}518 for ablations).
Median organism lifespan reveals an individual-vs-population distinction:
reproduction-ablated organisms live longer individually (median 751 vs.\
132 steps for normal) despite population collapse, confirming that the
population decline reflects absent replacement rather than individual
fragility.
Per-seed distributions are shown in Figure~\ref{fig:distributions}.

At $T=500$ (before long-term population dynamics dominate),
reproduction-ablated organisms retain higher survival
($\bar{x}$=47.7 vs.\ 31.3 at $T=2000$; normal baseline
$\bar{x}_{\text{normal}}$=153.5 at $T=500$), with 69\% decline
relative to normal versus 89\% at $T=2000$.
This progressive widening confirms that the long-term effect is
population-level (no replacement) rather than immediate individual-level
failure---organisms can self-maintain without reproduction.

\begin{figure}[t]
\centering
\IfFileExists{figures/fig_distributions.pdf}%
  {\includegraphics[width=\columnwidth]{figures/fig_distributions.pdf}}%
  {\fbox{\parbox{\dimexpr\columnwidth-2\fboxsep-2\fboxrule}{%
    \centering\vspace{2em}[fig\_distributions.pdf --- to be generated]\vspace{2em}}}}
\caption{Per-seed distributions of final alive count ($N_T$) across all
conditions ($n$=30 per condition). Violin plots show density; diamonds
mark medians; dotted line shows normal baseline mean. The distribution
confirms that ablation effects are consistent across seeds, not driven
by outliers.}
\label{fig:distributions}
\end{figure}

\begin{figure*}[t]
\centering
\includegraphics[width=\textwidth]{figures/fig_timeseries.pdf}
\caption{Population dynamics under criterion ablation. Lines show mean alive
count across 30 seeds (100--129); shaded bands show $\pm$1 SEM. Normal
baseline (thick black) stabilizes near 293 organisms by step~1900. Removing reproduction,
response, or metabolism causes $>$84\% population collapse. Evolution ablation
shows a modest 5\% decline (Cliff's $\delta$=0.32), consistent with optimization
rather than short-term survival necessity.}
\label{fig:timeseries}
\end{figure*}

\paragraph{Functional analogy verification.}
For each criterion, all three conditions are satisfied:
(a)~each consumes resources per step (energy for boundary repair, metabolic
computation, NN evaluation);
(b)~ablation causes significant degradation (Table~\ref{tab:ablation}); and
(c)~feedback loops are observable (e.g., metabolism~$\leftrightarrow$~boundary:
energy funds repair, boundary collapse stops metabolism).
Thus, each criterion qualifies as a functional analogy, not a simplified proxy.

\paragraph{Quantitative coupling evidence.}
Time-lagged cross-correlation of per-step population means under normal
conditions confirms the coupling pathways in Table~\ref{tab:criteria}.
Energy and boundary integrity show strong anti-correlation
($r=-0.85$, $p<0.001$, lag~0): repair continuously converts energy
into boundary maintenance ($\Delta b_{\text{repair}} \propto e$),
so higher boundary integrity corresponds to depleted energy reserves.
Energy and internal state $s_0$ show strong positive correlation
($r=0.81$, $p<0.001$, lag~3 steps), confirming that metabolic state
drives homeostatic regulation with a short delay.
Boundary integrity and $s_0$ show moderate correlation
($r=-0.55$, $p<0.001$, lag~3), consistent with the
homeostasis$\to$boundary pathway operating through repair efficacy.

\subsection{Proxy Control Comparison}

To test whether criterion ablation merely reflects tautological
definitions, we compare three metabolism implementations on the same seeds
($n$=30): \textbf{Counter} (minimal single-step conversion, no waste),
\textbf{Toy} (single-step with waste dynamics), and \textbf{Graph} (full
multi-step network with catalytic nodes).
All three satisfy ``dynamic resource consumption,'' yet produce
qualitatively different dynamics (Figure~\ref{fig:proxy}).
Graph metabolism supports the highest genome diversity (7.58 vs.\ 5.69 for
Toy) despite sustaining fewer organisms (293 vs.\ 322 for Toy, 374 for
Counter), indicating that metabolic complexity imposes greater selective
pressure.
This confirms that the specific implementation---not merely the presence---of
a criterion shapes system behavior, ruling out tautological definitions.

\begin{figure}[t]
\centering
\IfFileExists{figures/fig_proxy.pdf}%
  {\includegraphics[width=\columnwidth]{figures/fig_proxy.pdf}}%
  {\fbox{\parbox{\dimexpr\columnwidth-2\fboxsep-2\fboxrule}{%
    \centering\vspace{2em}[fig\_proxy.pdf --- to be generated]\vspace{2em}}}}
\caption{Proxy control comparison. Three metabolism implementations of
increasing complexity on the same seeds ($n$=30). Graph metabolism sustains
fewer organisms but higher genome diversity and waste dynamics, demonstrating
that metabolic complexity produces qualitatively distinct ecological dynamics
rather than simply increasing population counts.}
\label{fig:proxy}
\end{figure}

\subsection{Pairwise Ablations and Interdependence}

To test for interaction effects beyond individual necessity, we disable
pairs of criteria and compute synergy:
$\text{synergy}_{A,B} = \Delta_{A \cup B} - (\Delta_A + \Delta_B)$.
Table~\ref{tab:pairwise} reports scores for six pairs.
All show sub-additive synergy (negative), indicating \emph{shared failure
pathways}: individual ablations already collapse populations to near
their floor ($\sim$30--50 organisms), leaving no room for additive effects.
This ceiling effect reveals that criteria damage overlapping subsystems.
The (metabolism, homeostasis) pair exemplifies the shared pathway:
disabling metabolism eliminates energy production, starving boundary
repair ($\Delta b_{\text{repair}} \propto e$); disabling homeostasis
degrades adaptive behavior, accelerating waste accumulation, which
amplifies boundary decay via the waste-pressure term ($s_w \cdot w$
in $\Delta b_{\text{decay}}$).
Both routes converge on boundary failure, explaining why
$\Delta_{AB} = 249.9$ falls far below the additive expectation of 472.0.

\begin{table}[t]
\centering
\caption{Pairwise ablation synergy scores ($n$=30 per condition, Graph
metabolism). Negative synergy indicates sub-additive interaction (shared
failure pathways). Baseline $\bar{x}$=293.1, consistent with
Table~\ref{tab:ablation}.}
\label{tab:pairwise}
\small
\begin{tabular}{@{}l@{\hskip 5pt}r@{\hskip 5pt}r@{\hskip 5pt}r@{\hskip 5pt}r@{}}
\toprule
Pair & $\Delta_{AB}$ & Exp. & Syn. & Ratio \\
\midrule
(Metab, Homeo)     & 249.9 & 472.0 & $-$222.0 & 0.53 \\
(Metab, Resp.)     & 247.7 & 506.1 & $-$258.4 & 0.49 \\
(Reprod, Growth)   & 261.9 & 369.4 & $-$107.5 & 0.71 \\
(Bdry, Homeo)      & 181.9 & 396.2 & $-$214.3 & 0.46 \\
(Resp., Homeo)     & 256.4 & 483.9 & $-$227.5 & 0.53 \\
(Reprod, Evol)     & 261.9 & 276.7 & $-$14.9  & 0.95 \\
\bottomrule
\multicolumn{5}{@{}l@{}}{\footnotesize Exp.\ = $\Delta_A{+}\Delta_B$; Ratio = $\Delta_{AB}$/Exp.}
\end{tabular}
\end{table}

\paragraph{Growth--reproduction confound.}
The (reproduction, growth) pair confirms: $\Delta_{AB} \approx
\Delta_{\text{reproduction}}$ = 261.9, so growth's effect is fully mediated
through reproduction gating.
This clarifies growth's role as an upstream prerequisite for reproductive
competence rather than an independent viability mechanism.

\subsection{Evolution Strengthening}

The modest single-ablation effect of evolution ($d$=0.57, Cliff's
$\delta$=0.32) reflects the 2000-step simulation horizon.
To demonstrate evolution's contribution at longer timescales, we run two
additional experiments:

\paragraph{Long run.}
Extending simulations to 10,000 steps ($n$=30) yields $d$=1.43, Cliff's
$\delta$=0.72 ($p < 0.001$), a large effect compared to the modest
2000-step result.
Normal populations reach $\bar{x}$=358.1 organisms versus 327.1 for
no-evolution ($\Delta$=$-$8.7\%), confirming that evolutionary adaptation
accumulates across generations.

\paragraph{Environmental shift.}
At step 2,500 of a 5,000-step simulation, resource regeneration rate is
halved (from 0.01 to 0.005 per cell per step).
Evolved populations recover more effectively ($\bar{x}$=384.6 vs.\ 361.4,
$d$=1.01, Cliff's $\delta$=0.57, $p < 0.001$), demonstrating that
evolution enables adaptive response to environmental perturbation.

\begin{figure}[t]
\centering
\IfFileExists{figures/fig_evolution.pdf}%
  {\includegraphics[width=\columnwidth]{figures/fig_evolution.pdf}}%
  {\fbox{\parbox{\dimexpr\columnwidth-2\fboxsep-2\fboxrule}{%
    \centering\vspace{2em}[fig\_evolution.pdf --- to be generated]\vspace{2em}}}}
\caption{Evolution strengthening. Top: 10,000-step long run shows
increasing divergence between normal and no-evolution conditions.
Bottom: environmental shift at step 2,500 (dashed line) demonstrates
evolved populations' superior recovery.
Lines show mean across 30 seeds; shaded bands $\pm$1 SEM.}
\label{fig:evolution}
\end{figure}

\subsection{Homeostatic Regulation}

Figure~\ref{fig:homeostasis} shows population-mean internal state
trajectories for Normal versus No Homeostasis conditions.
Under normal operation, the neural-network controller actively regulates
internal state variable~$s_0$, maintaining it near 0.99 throughout
the simulation (Panel~A, solid).
When homeostasis is disabled, $s_0$ decays monotonically due to the
passive decay term ($h_{\text{decay}} = 0.01$ per step), and the
population exhibits high variance as organisms lack the ability to
counteract perturbations (Panel~A, dashed; variance shown in Panel~B).
This divergence confirms that the NN-driven homeostatic mechanism
constitutes an active regulatory process---not a static parameter---that
satisfies the functional-analogy requirement of sustained resource
consumption with measurable degradation upon removal.

\begin{figure}[t]
\centering
\IfFileExists{figures/fig_homeostasis.pdf}%
  {\includegraphics[width=\columnwidth]{figures/fig_homeostasis.pdf}}%
  {\fbox{\parbox{\dimexpr\columnwidth-2\fboxsep-2\fboxrule}{%
    \centering\vspace{2em}[fig\_homeostasis.pdf --- to be generated]\vspace{2em}}}}
\caption{Homeostatic regulation trajectories ($n$=30). (A)~Mean internal
state variable~$s_0$ over time: Normal condition (solid) maintains
regulation near 0.99, while No Homeostasis (dashed) shows passive decay.
(B)~Population-level standard deviation of $s_0$: disabled homeostasis
produces higher inter-organism variance. Shaded bands show $\pm$1 SEM.}
\label{fig:homeostasis}
\end{figure}

\subsection{Graded Metabolic Ablation}

To test whether criterion ablation effects reflect a continuous
functional relationship rather than a binary on/off artifact, we sweep
the metabolism efficiency multiplier over $\{1.0, 0.75, 0.5, 0.25, 0.0\}$
($n$=30 per level, 1,000 steps, graph metabolism).
Figure~\ref{fig:graded} shows a monotonic dose-response: population
viability decreases proportionally with metabolic efficiency
(Jonckheere-Terpstra trend test, $p < 0.001$).
This graded response demonstrates that metabolic contribution is
\emph{quantitatively proportional}, not merely a binary switch---partial
impairment produces proportional degradation, consistent with genuine
functional analogy.

\begin{figure}[t]
\centering
\IfFileExists{figures/fig_graded.pdf}%
  {\includegraphics[width=0.48\columnwidth]{figures/fig_graded.pdf}}%
  {\fbox{\parbox{0.45\columnwidth}{%
    \centering\vspace{2em}[fig\_graded.pdf]\vspace{2em}}}}
\caption{Graded metabolic ablation dose-response. Median final alive
count (line) with IQR (shaded) across 30 seeds per level.
Monotonic decline confirms continuous functional dependency
($p < 0.001$, Jonckheere-Terpstra).}
\label{fig:graded}
\end{figure}

\subsection{Cyclic Environment}

To assess whether evolved populations exhibit adaptive resilience beyond
static conditions, we subject organisms to periodic resource modulation
(period~=~2,000 steps; high phase: 0.01, low phase: 0.005 resource
per cell per step) over 10,000 steps ($n$=30).
Figure~\ref{fig:cyclic} compares evolution-on versus evolution-off
populations across five complete resource cycles.
Evolved populations ($\bar{x}$=336.0) significantly outperform
non-evolving populations ($\bar{x}$=312.6) across all cycles
($d$=1.10, Cliff's $\delta$=0.58, $p < 0.001$), showing faster
recovery after each low-resource phase.
This demonstrates that evolutionary adaptation provides measurable
resilience to \emph{recurring} environmental stress, not just one-time
perturbation.

\begin{figure}[t]
\centering
\IfFileExists{figures/fig_cyclic.pdf}%
  {\includegraphics[width=0.48\columnwidth]{figures/fig_cyclic.pdf}}%
  {\fbox{\parbox{0.45\columnwidth}{%
    \centering\vspace{2em}[fig\_cyclic.pdf]\vspace{2em}}}}
\caption{Cyclic environment test. Population dynamics under periodic
resource modulation (period~=~2,000 steps). Lightly shaded bands
mark low-resource phases. Evolution-on populations recover faster
after each stress phase. Lines: mean across 30 seeds; bands: $\pm$1 SEM.}
\label{fig:cyclic}
\end{figure}

\paragraph{Sham ablation control.}
To confirm that observed ablation effects are functional rather than
computational artifacts, we implement a state-neutral sham process that
performs spatial neighbor queries matching the computational cost of a
real criterion update but discards all results without consuming RNG
state or modifying simulation variables.
Comparing sham-on versus sham-off ($n$=30, 1,000 steps) yields no
significant difference (Mann-Whitney $U$, $p > 0.05$), validating
that criterion ablation effects reflect genuine functional dependencies.

\subsection{Phenotype Clustering}

Clustering analysis of per-seed organism traits (energy, waste, boundary
integrity, genome diversity, generation count) reveals emergent phenotypic
differentiation among evolved populations (Figure~\ref{fig:phenotype}).
This suggests that the evolutionary dynamics of our system produce
behaviorally distinct organism strategies---an emergent property not
explicitly engineered into the criteria implementation.

\begin{figure}[t]
\centering
\IfFileExists{figures/fig_phenotype.pdf}%
  {\includegraphics[width=0.48\columnwidth]{figures/fig_phenotype.pdf}}%
  {\fbox{\parbox{0.45\columnwidth}{%
    \centering\vspace{2em}[fig\_phenotype.pdf]\vspace{2em}}}}
\caption{Phenotype clustering via PCA projection of organism trait vectors.
Clusters identified by $k$-means (optimal $k$ selected by silhouette score)
suggest emergent behavioral strategies among evolved populations.}
\label{fig:phenotype}
\end{figure}

% ============================================================================
\section{Discussion}

\paragraph{Criterion interdependence.}
Single ablations reveal a hierarchy: reproduction, response, and metabolism
form an essential triad ($>$84\% decline), while homeostasis and boundary
occupy a middle tier ($\sim$58--77\%).
Pairwise ablations show uniformly sub-additive interactions
(Table~\ref{tab:pairwise}), consistent with \emph{shared failure pathways}
rather than independent modules---criteria converge on overlapping viability
subsystems.
The proxy control comparison further demonstrates that the specific
implementation of a criterion shapes ecological dynamics (graph metabolism
produces higher diversity but lower populations than simpler alternatives),
ruling out tautological definitions.

\paragraph{Evolution at longer timescales.}
The modest 2000-step evolution effect ($d$=0.57, $\delta$=0.32) grows to
$d$=1.43 ($\delta$=0.72) at 10,000 steps and $d$=1.01 ($\delta$=0.57)
under environmental perturbation, confirming that evolutionary adaptation
accumulates across generations.
The cyclic environment test (Figure~\ref{fig:cyclic}) further demonstrates
that evolved populations recover more rapidly from recurring resource
stress, indicating adaptive resilience beyond one-time environmental shifts.

\paragraph{Are criteria merely engineered?}
One might argue that criterion ablation merely reflects system
design---disabling any engineered subsystem would degrade performance.
Four lines of evidence argue against this interpretation.
First, the proxy control comparison shows that alternative metabolism
implementations satisfying the same functional-analogy definition produce
qualitatively distinct ecological dynamics (Figure~\ref{fig:proxy}),
ruling out tautological definitions.
Second, pairwise ablations reveal sub-additive interactions
(Table~\ref{tab:pairwise}), indicating shared failure pathways
inconsistent with independent modules.
Third, the state-neutral sham ablation control (matching real
criterion computational cost without modifying simulation state) produces
no significant population difference, confirming that observed effects
are functional, not computational artifacts.
Fourth, graded metabolic ablation produces a proportional dose-response
(Figure~\ref{fig:graded}), inconsistent with a binary on/off artifact.

\paragraph{Criterion maturity.}
Criteria are at varying levels of implementation maturity: five criteria
(cellular organization, metabolism, homeostasis, reproduction, response)
reach Level~4 (multi-process interaction), while growth and evolution
are at Level~3 (dynamic single process).
Evolution demonstrates heritable variation and differential survival,
sufficient for functional analogy though not open-ended evolution (Level~5).

\subsection{Limitations}

\textbf{Cellular organization} is tracked via a scalar boundary-integrity
variable rather than emergent spatial cohesion among swarm agents;
an explicit spatial boundary model would provide a stronger functional
analogy to biological membranes.
\textbf{Growth} uses a maturation toggle that functions primarily as a
reproduction gate rather than a full developmental program;
a morphogenetic model would strengthen this criterion's
functional-analogy claim.
\textbf{Evolution} reaches $d$=1.43 at 10,000 steps but demonstrating
open-ended dynamics would require $10^5$+ steps with novelty metrics
\citep{bedau_2000_open}.
\textbf{Scale} is limited to $\sim$300 organisms on a single machine;
larger populations might reveal emergent ecological phenomena.
We adopt a \textbf{weak ALife} stance: this is a functional model, not
a claim that digital organisms are alive.

% ============================================================================
\section{Conclusion}

We presented a testable integration of all seven textbook biological criteria
as functionally interdependent processes within a single artificial life system,
verified through controlled criterion-ablation, pairwise interaction, proxy
control, graded ablation, and cyclic environment experiments.
The functional-analogy framework---requiring dynamic operation, measurable
degradation upon removal, and feedback coupling---provides a rigorous
standard distinguishing genuine criteria implementations from simplified
proxies.

Our results demonstrate that no criterion is decorative: removing any one
causes statistically significant population decline ($p < 0.016$, Holm-Bonferroni
corrected), with Cliff's $\delta$ ranging from 0.32 to 1.00.
Pairwise ablations further reveal shared failure pathways---sub-additive
interactions consistent with criteria converging on overlapping viability
subsystems rather than operating independently.

Future work will pursue three directions:
(1)~a richer developmental program replacing the current growth toggle;
(2)~scaling to larger populations to investigate emergent ecological phenomena
and open-ended evolution metrics \citep{bedau_2000_open,
taylor_2016_openended}; and
(3)~systematic environmental perturbation studies to characterize adaptive
capacity across evolutionary timescales.

Code and data will be made available upon acceptance at an anonymous repository.

\bibliographystyle{apalike}
\bibliography{references}

\end{document}
