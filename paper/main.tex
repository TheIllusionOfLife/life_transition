\documentclass[letterpaper]{article}

\usepackage{natbib,alifeconf}
\usepackage{graphicx,amsmath,amssymb,booktabs,xcolor,multirow}
\usepackage{url,hyperref}

\newcommand\blfootnote[1]{%
  \begingroup
  \renewcommand\thefootnote{}\footnote{#1}%
  \addtocounter{footnote}{-1}%
  \endgroup
}

% Conditional figure include: show PDF if present, otherwise a labelled placeholder box.
% No \fbox used for the placeholder — avoids overfull hbox from fboxsep/fboxrule padding.
\newcommand{\figifexists}[4]{%   #1=path  #2=width  #3=options  #4=alt label
  \IfFileExists{#1}{%
    \includegraphics[width=#2,#3]{#1}%
  }{%
    \begin{minipage}{#2}\vspace{2em}\centering\footnotesize[Fig.\ pending: #4]\vspace{2em}\end{minipage}%
  }%
}

\title{Semi-Life: A Capability Ladder for the Virus-to-Life Transition}

\author{Anonymous}

\begin{document}
\maketitle

% ============================================================================
\begin{abstract}
When does a minimal replicator become life-like?
We introduce \emph{Semi-Life}: a parameterized family of minimal
replicators---Viroid, Virus, and ProtoOrganelle---cohabiting a seven-criterion
artificial life world \citep{background_2026_digital}.
Each archetype begins at a different point on a \emph{Capability Ladder}
(V0: replication; V1: boundary; V2: homeostasis; V3: metabolism;
V4: response to stimuli; V5: staged lifecycle) and gains capabilities one step
at a time.
V1 boundary protects against energy leakage and environmental damage;
V2 homeostasis mitigates overconsumption waste.
Progress is quantified by a multi-channel \emph{Internalization Index}~(II)
averaging energy, regulation, behaviour, and lifecycle channels---a continuous
axis from virus-like ($\mathrm{II}\approx0$) to life-like ($\mathrm{II}>0$).
Eight directional hypotheses~(H1--H8) were pre-registered with
Holm-Bonferroni correction across 32~tests.
Using $n\!=\!100$ held-out test seeds across a four-level resource harshness
axis, we find environment-dependent cost--benefit tradeoffs: V1 reduces
survival in resource-rich environments but protects in harsh ones,
V2 overconsumption regulation provides measurable benefit,
V3 metabolism produces a dramatic recovery, and
V4~chemotaxis significantly improves survival in resource-scarce
environments by enabling gradient-following movement.
V5 lifecycle staging provides dormancy-based energy conservation.
The full V0$\to$V5 ladder demonstrates that the virus-to-life transition
is a cost-then-benefit trajectory where each capability addition is
justified only when coupled with sufficient internal infrastructure.
\end{abstract}

Submission type: \textbf{Full Paper}\\

\blfootnote{\textcopyright 2026 The Author(s). Published under a Creative
Commons Attribution 4.0 International (CC BY 4.0) license.}

% ============================================================================
\section{Introduction}
\label{sec:intro}

The question ``what is life?'' resists clean philosophical resolution.
Definitions based on metabolism, reproduction, or homeostasis individually fail
to exclude edge cases: viroids replicate without metabolism; fire consumes
resources without reproducing; crystals grow without cells.
\citet{cleland_2002_defining} argue that any list-based definition risks
circularity or counter-example, while \citet{benner_2010_defining} note that
borderline cases---viroids, viruses, prions, obligate-intracellular
parasites---are precisely where a definition is most needed.

A complementary approach replaces the question ``is it alive?'' with a
measurable continuum: \emph{to what degree does this entity exhibit
life-like functional organisation?}
The functional analogy framework of \citet{background_2026_digital} operationalises
this question for ALife systems: a capability is a functional analogy of a
biological criterion if and only if (a) it requires sustained resource
consumption, (b) its removal causes measurable population degradation, and
(c) it forms a feedback loop with at least one other criterion.
Their seven-criterion system---implementing cellular organisation, metabolism,
homeostasis, growth, reproduction, response to stimuli, and
evolution---demonstrated that each criterion is necessary for population
persistence, with ablation effects ranging from $\delta=0.39$ to $1.00$
(Cliff's~$\delta$, Holm-Bonferroni corrected).

The present work asks a complementary question: \emph{can a minimal replicator
become life-like by internalising the functions it initially outsources?}
We introduce \textbf{Semi-Life}---three archetypal minimal replicators (Viroid,
Virus, ProtoOrganelle) that inhabit the same seven-criterion world---and equip
each with a \emph{Capability Ladder} from bare replication~(V0) to internal
metabolism~(V3).
An \emph{Internalization Index}~(II) tracks the fraction of each entity's energy
budget that comes from internal conversion rather than direct environmental
uptake, providing a continuous axis from virus-like ($\mathrm{II}=0$) to
life-like ($\mathrm{II}>0$).

We pre-registered eight hypotheses (H1--H8) covering distinct aspects of the
transition---boundary cost--benefit tradeoff, metabolic buffering, replication
liberation, monotonic capability trends, chemotaxis benefit, lifecycle staging,
the full V0$\to$V5 monotonic trend, and overconsumption regulation---and tested
them on $n=100$ held-out seeds across a four-level resource harshness axis.
Our contributions are: (i) an operational, replicable protocol for measuring the
virus-to-life transition using a pre-existing ALife world as the background
platform; (ii) phase diagrams showing where in the capability~$\times$~harshness
space survival boundaries shift; (iii) confirmation or disconfirmation of all eight
pre-registered directional predictions; and (iv) the InternalizationIndex as a
falsifiable metric of life-likeness progress.

% ============================================================================
\section{Background Platform}
\label{sec:background}

The background world is the seven-criterion ALife system described in
\citet{background_2026_digital}, which we summarise here.
A population of \emph{organisms} inhabits a continuous 2D environment.
Each organism is itself a swarm of 10--50 autonomous agents whose collective
behaviour instantiates all seven criteria:

\begin{enumerate}
\item \textbf{Cellular organisation.}
  Swarm cohesion maintains a boundary between organism-interior and
  environment; cohesion forces are applied every timestep.
\item \textbf{Metabolism.}
  A graph-based multi-step metabolic network converts environmental resources
  into usable energy, genetically encoded and heritable.
\item \textbf{Homeostasis.}
  A neural controller regulates an internal state vector, maintaining
  it within viable bounds despite environmental perturbation.
\item \textbf{Growth and development.}
  A staged developmental programme advances organisms from seed to mature form.
\item \textbf{Reproduction.}
  Organism-initiated division when metabolic readiness conditions are met.
\item \textbf{Response to stimuli.}
  Local sensory input drives neural-network-mediated action selection.
\item \textbf{Evolution.}
  Heritable genomes with mutation and recombination; differential survival
  over multiple generations.
\end{enumerate}

Each criterion satisfies the functional analogy conditions:
(a)~it costs resources every timestep, (b)~its mid-simulation removal causes
statistically significant population decline (all $p_\text{corr}\!\leq\!0.005$,
Holm-Bonferroni, Mann-Whitney~U, $n\!=\!30$ per condition), and (c)~it
participates in at least one cross-criterion feedback loop measured by
lagged correlation.

Critically, the world's organism population is \emph{not affected} by the
Semi-Life entities introduced in this paper: SemiLife uptake
(0.02/step for 10--50 entities) is negligible relative to organism
consumption, so the organism population dynamics are effectively
unchanged.
The background world provides (i)~a structured resource landscape with
spatial heterogeneity driven by organism activity, and (ii)~temporal
dynamics from organism metabolism and reproduction that create a
realistic, non-stationary environment for the minimal replicators.
SemiLife entities do not directly interact with organisms.

Initial entity positions are drawn uniformly at random from the world
grid, seeded deterministically from the world seed.
The resource field is initialised on a 100$\times$100 uniform grid.

% ============================================================================
\section{The Semi-Life Model}
\label{sec:model}

\subsection{Archetypes}

Three archetypal parameterisations represent different ``starting points'' on
the life-likeness axis, motivated by their biological counterparts
\citep{urry_2020_campbell}:

\begin{itemize}
\item \textbf{Viroid} ($\approx$~naked RNA): baseline V0 only (pure
  replication, no boundary or regulation).
  Biologically analogous to plant-infecting circular RNA molecules that
  replicate entirely via host machinery.
\item \textbf{Virus} ($\approx$~capsid-enclosed genome): baseline V0$+$V1
  (replication plus boundary integrity).
  Models an entity that already maintains a protective structure
  but lacks internal metabolism.
\item \textbf{ProtoOrganelle} ($\approx$~proto-endosymbiont): baseline
  V1$+$V2$+$V3 without V0.
  Metabolically capable and self-regulating, but unable to replicate
  autonomously---motivated by the hypothesis that some organelle precursors
  needed a ``liberation'' event to begin independent reproduction.
\end{itemize}

Key archetype parameters are summarised in Table~\ref{tab:archetypes}.
Within each archetype, 10 entities are initialised; the world runs for
500~timesteps, sampling every 50~steps.

\begin{table}[t]
\centering
\footnotesize
\caption{Archetype parameter summary.
  Shared parameters: \texttt{maintenance\_cost}~$=0.0005$,
  \texttt{resource\_uptake\_rate}~$=0.02$,
  \texttt{internal\_conversion\_rate}~$=0.05$.
  V1/V2 protective parameters and V4/V5 parameters are shared across
  all archetypes.}
\label{tab:archetypes}
\resizebox{\columnwidth}{!}{%
\begin{tabular}{lccc}
\toprule
Parameter & Viroid & Virus & ProtoOrganelle \\
\midrule
Baseline capabilities & V0 & V0+V1 & V1+V2+V3 \\
\texttt{replication\_threshold} & 0.60 & 0.60 & 0.80 \\
\texttt{replication\_cost} & 0.27 & 0.27 & 0.30 \\
\texttt{boundary\_decay\_rate} & 0.002 & 0.001 & 0.002 \\
\texttt{boundary\_repair\_rate} & 0.010 & 0.010 & 0.010 \\
\midrule
\multicolumn{4}{l}{\emph{V1 protective parameters (shared)}} \\
\texttt{energy\_leakage\_rate} & \multicolumn{3}{c}{0.005} \\
\texttt{env\_damage\_probability} & \multicolumn{3}{c}{0.05} \\
\texttt{env\_damage\_amount} & \multicolumn{3}{c}{0.05} \\
\texttt{boundary\_damage\_absorption} & \multicolumn{3}{c}{0.8} \\
\texttt{boundary\_damage\_integrity\_cost} & \multicolumn{3}{c}{0.02} \\
\midrule
\multicolumn{4}{l}{\emph{V2 overconsumption parameters (shared)}} \\
\texttt{overconsumption\_waste\_fraction} & \multicolumn{3}{c}{0.3} \\
\texttt{optimal\_uptake\_rate} & \multicolumn{3}{c}{0.015} \\
\midrule
\multicolumn{4}{l}{\emph{V4 chemotaxis parameters (shared)}} \\
\texttt{v4\_move\_cost} & \multicolumn{3}{c}{0.01} \\
\texttt{v4\_max\_speed} & \multicolumn{3}{c}{1.0} \\
\texttt{v4\_mutation\_sigma} & \multicolumn{3}{c}{0.05} \\
\midrule
\multicolumn{4}{l}{\emph{V5 lifecycle parameters (shared)}} \\
\texttt{v5\_activation\_threshold} & \multicolumn{3}{c}{0.6} \\
\texttt{v5\_dispersal\_age} & \multicolumn{3}{c}{100} \\
\texttt{v5\_dispersal\_duration} & \multicolumn{3}{c}{20} \\
\texttt{v5\_dormant\_decay\_mult} & \multicolumn{3}{c}{0.3} \\
\texttt{v5\_dispersal\_decay\_mult} & \multicolumn{3}{c}{1.5} \\
\texttt{v5\_dispersal\_speed\_mult} & \multicolumn{3}{c}{2.0} \\
\bottomrule
\end{tabular}%
}
\end{table}

\subsection{Capability Ladder}

Capabilities are encoded as a bitmask.
All capabilities are \emph{dynamic processes} satisfying functional analogy
condition~(a): they consume resources every timestep.

\textbf{V0---Replication} (bit~0x01).
When \texttt{maintenance\_energy} $\geq$ \texttt{replication\_threshold},
the entity pays \texttt{replication\_cost} and spawns a copy within
\texttt{replication\_spawn\_radius}.
Without V0, no new copies can be created regardless of energy state.
Entities die when \texttt{maintenance\_energy} reaches zero; without
capabilities, a V0-only entity relies entirely on external resource uptake
to offset the per-step \texttt{maintenance\_cost}.

\textbf{V1---Boundary integrity} (bit~0x02).
A scalar \texttt{boundary\_integrity} $\in [0,1]$ decays by
\texttt{boundary\_decay\_rate} per timestep and is actively repaired toward~$1$
at \texttt{boundary\_repair\_rate}.
If integrity falls below \texttt{boundary\_death\_threshold}~$=0.1$, the entity
dies; replication is blocked below \texttt{boundary\_replication\_min}~$=0.5$.

V1 provides two protective benefits.
First, entities \emph{without} V1 lose energy through diffusion-like leakage
at \texttt{energy\_leakage\_rate}~$=0.005$ per step, modelling thermodynamic
dissipation in the absence of a membrane.
Second, stochastic environmental damage events occur with probability
\texttt{env\_damage\_probability}~$=0.05$ per step; without V1, the full
\texttt{env\_damage\_amount}~$=0.05$ is deducted from energy.
With V1, the boundary absorbs a fraction
(\texttt{boundary\_damage\_absorption}~$=0.8 \times \text{integrity}$) of the
damage, at a cost to boundary integrity
(\texttt{boundary\_damage\_integrity\_cost}~$=0.02$).
This creates an environment-dependent tradeoff: in resource-rich conditions
where leakage and damage pressure are low relative to repair costs, V1 is a
net cost; in harsh environments where damage is proportionally more costly,
V1 provides a net benefit.

\textbf{V2---Homeostatic regulation} (bit~0x04).
A \texttt{regulator} state $\in [0,1]$ scales the resource uptake rate
proportionally, implementing a demand-side throttle.
The regulator costs \texttt{regulator\_cost\_per\_step}~$=0.0005$ per step.
Beyond throttling, V2 mitigates overconsumption waste: when resource uptake
exceeds \texttt{optimal\_uptake\_rate}~$=0.015$ per step, the excess incurs a
waste penalty of \texttt{overconsumption\_waste\_fraction}~$=0.3$.
With V2, the regulator reduces this waste by up to 80\%
($1 - 0.8 \times \text{regulator\_state}$), modelling feedback inhibition
analogous to metabolic pathway regulation in biochemistry.

\textbf{V3---Internal metabolism} (bit~0x08).
An \texttt{internal\_pool} stores resources and converts them to maintenance
energy at \texttt{internal\_conversion\_rate}~$=0.05$ per step.
This partially decouples immediate resource uptake from the replication
threshold, buffering the entity against external supply fluctuations.

\textbf{V4---Response to stimuli} (bit~0x10).
A linear policy $\mathbf{w} \in \mathbb{R}^8$ maps a sensory input vector
$\mathbf{s} = [\nabla_x r, \nabla_y r, e, b, n, a, s_6, s_7]$---where
$\nabla r$ is the local resource gradient, $e$ is normalised energy,
$b$ is boundary integrity, $n$ is neighbour density, and $a$ is normalised
age---to a velocity offset $\Delta\mathbf{v} = \mathbf{w} \cdot \mathbf{s}$,
clamped to $\pm$\texttt{v4\_max\_speed}~$=1.0$.
Each movement step costs \texttt{v4\_move\_cost}~$=0.01 \times |\Delta\mathbf{v}|$.
Offspring inherit the parental policy with per-weight Gaussian noise
($\sigma\!=\!0.05$), enabling heritable variation.
Default initialisation ($w_1\!=\!w_2\!=\!0.5$, others zero) produces
gradient-following chemotaxis.
Dimensions $s_6$ and $s_7$ are reserved for future sensory extensions
(e.g., resource memory, social signals) and are currently zero.

\textbf{V5---Staged lifecycle} (bit~0x20).
Entities cycle through three stages with distinct behavioural multipliers:
\emph{Dormant} (energy decay $\times 0.3$, no replication, no movement---hibernation),
\emph{Active} (all multipliers $\times 1.0$---normal operation), and
\emph{Dispersal} (decay $\times 1.5$, speed $\times 2.0$, no replication---fast
dispersal at metabolic cost).
Transitions are deterministic: Dormant$\to$Active when energy exceeds
\texttt{v5\_activation\_threshold}~$=0.6$; Active$\to$Dispersal after
\texttt{v5\_dispersal\_age}~$=100$ ticks; Dispersal$\to$Dormant after
\texttt{v5\_dispersal\_duration}~$=20$ ticks or when energy falls below~$0.2$.

\subsection{Multi-Channel InternalizationIndex}

The Internalization Index is a composite of four per-step channels, each
measuring how much a specific function is internally rather than externally
driven.
All accumulators reset at the start of each step.

\begin{itemize}
\item \textbf{Energy} (V3): $\mathrm{II}_E = E_\text{int}/(E_\text{int}+E_\text{ext})$,
  where $E_\text{int}$ is energy from internal pool conversion and
  $E_\text{ext}$ from direct resource uptake.
\item \textbf{Regulation} (V2): $\mathrm{II}_R$, the fraction of
  overconsumption waste saved by the regulator relative to unregulated
  uptake.
  Measures how much V2 contributes to energy efficiency.
\item \textbf{Behaviour} (V4): $\mathrm{II}_B = |\Delta\mathbf{v}|/v_\text{max}$,
  the fraction of maximum possible movement that is policy-driven.
  Measures behavioural autonomy.
\item \textbf{Lifecycle} (V5): $\mathrm{II}_L$, whether stage transitions
  are driven by internal state conditions (energy thresholds, age).
  Measures lifecycle self-control.
\end{itemize}

The composite Internalization Index is the mean of active channels:
\begin{equation}
  \mathrm{II} = \frac{1}{|C|}\sum_{c \in C} \mathrm{II}_c
  \quad\text{where } C = \{c : \text{total}_c > \epsilon\}.
  \label{eq:ii}
\end{equation}
V0-only entities have $|C|=0$ and $\mathrm{II}=0$.
As capabilities are added, new channels activate: V2 adds regulation,
V3 adds energy, V4 adds behaviour, V5 adds lifecycle.
This ensures II rises \emph{gradually} across the ladder rather than as a
V3 step function.
Per-channel values are reported alongside the composite to enable
independent validation.
Crucially, \emph{survival metrics} (alive count, total replications) are
computed from entity counts, not from~$\mathrm{II}$, eliminating circularity.

% ============================================================================
\section{Experiments}
\label{sec:experiments}

\subsection{Resource Harshness Axis}

The resource field is initialised with \texttt{resource\_initial\_value}
$\in \{1.0, 0.3, 0.1, 0.05\}$ (labelled Rich, Medium, Sparse, Scarce).
A lower initial value reduces the total resource pool from 10\,000 to 500 units
(100$\times$100 grid).
The regeneration rate is fixed at 0.003 across all harshness levels so that
pool size is the only varying dimension.

\paragraph{Calibration objective.}
Parameters were calibrated on seeds 0--99 targeting non-trivial population
dynamics (neither universal extinction nor unbounded growth) across all four
harshness levels.
Specific criteria: (i)~V0-only Viroid achieves $\sim$50--100 alive at step~500
in rich; (ii)~V0+V1 shows an environment-dependent tradeoff (hurts in rich,
helps or neutral in sparse/scarce); (iii)~V0+V1+V2 measurably better than
V0+V1 via waste reduction; (iv)~V3 remains the largest survival jump;
(v)~at least 50\% of H1--H8 comparisons yield $\delta \in [0.3, 0.95]$
(not all saturated); (vi)~multi-channel II increases at each V-level addition.

Each condition is run for 100 test seeds (seeds 100--199), producing
$13\ \text{conditions} \times 4\ \text{harshness} \times 100\ \text{seeds}
= 5{,}200$ runs.

\subsection{Shock Axis}

Shock resilience is evaluated under periodic resource crashes at sparse
harshness (\texttt{resource\_initial\_value}~$=0.1$).
The environment cycles between a high phase (normal regen rate~$=0.003$) and
a low phase (20\% of normal: $0.0006$) with periods of 200 and 50 steps.
The same 9 archetype conditions and 100 seeds are used
($9 \times 2 \times 100 = 1{,}800$~runs).
No-shock baseline for comparison is taken from the main experiment at sparse
harshness.

\subsection{Pre-registered Hypotheses}
\label{sec:prereg}

Eight directional hypotheses were pre-registered before any test-seed data
collection (see supplementary pre-registration, committed to the repository
at \url{[ANONYMOUS]}).
Hypotheses H1--H7 were filed in Amendments 1--2; H8 was added in Amendment~3
(filed after receiving reviews but before any new simulation data was collected
with the revised model).
Holm-Bonferroni correction is applied across all 32 tests
(H1--H3, H5--H6, H8: $4~\text{harshness} \times 6~\text{hypotheses} = 24$;
H4, H7: $4$ harshness levels each via Jonckheere-Terpstra trend test).

\vspace{0.4em}
\noindent\textbf{H1} (\emph{Boundary cost--benefit tradeoff}, all harshness):
Viroid V0$+$V1 produces a \emph{different} alive count at step~500 than
Viroid V0, with the direction depending on environment harshness.
\emph{Rationale}: V1 boundary repair costs per-step energy, but V1 also
protects against energy leakage and environmental damage.
In resource-rich environments where damage pressure is low relative to repair
costs, V1 is a net cost; in harsh environments, V1's protective benefit
dominates.

\vspace{0.2em}
\noindent\textbf{H2} (\emph{Metabolism buffering}, all harshness):
Viroid V0$+$V1$+$V2$+$V3 produces \emph{more} alive entities at step~500 than
V0$+$V1$+$V2 across all four harshness levels.
\emph{Rationale}: V3 internal pool decouples replication threshold from
instantaneous uptake, buffering against resource fluctuations.

\vspace{0.2em}
\noindent\textbf{H3} (\emph{Replication liberation}, all harshness):
ProtoOrganelle V0$+$V1$+$V2$+$V3 (liberated) shows \emph{more}
\texttt{total\_replications} at step~500 than V1$+$V2$+$V3 (baseline) at all
harshness levels.
\emph{Rationale}: Without V0, ProtoOrganelle cannot replicate regardless of
metabolic state; V0 addition activates replication capability already supported
by the existing V1--V3 infrastructure.

\vspace{0.2em}
\noindent\textbf{H4} (\emph{Monotonic trend V0--V3}, all harshness):
Alive count at step~500 increases monotonically across Viroid V0, V0$+$V1,
V0$+$V1$+$V2, V0$+$V1$+$V2$+$V3 (Jonckheere-Terpstra test
\citep{jonckheere_1954_distribution}).

\vspace{0.2em}
\noindent\textbf{H5} (\emph{Chemotaxis benefit}, all harshness):
Viroid V0..V4 produces \emph{more} alive entities at step~500 than V0..V3.
\emph{Rationale}: V4 gradient-following enables directed movement toward
resource-rich areas, improving energy intake in heterogeneous environments.

\vspace{0.2em}
\noindent\textbf{H6} (\emph{Lifecycle staging}, all harshness):
Viroid V0..V5 produces \emph{more} alive entities at step~500 than V0..V4.
\emph{Rationale}: V5 dormancy conserves energy during scarcity, and dispersal
enables colonisation of resource patches, both improving population persistence.

\vspace{0.2em}
\noindent\textbf{H7} (\emph{Full monotonic trend V0--V5}, all harshness):
Alive count at step~500 increases monotonically across the full six-level Viroid
ladder V0 through V0..V5 (Jonckheere-Terpstra test).

\vspace{0.2em}
\noindent\textbf{H8} (\emph{V2 overconsumption regulation}, all harshness,
Amendment~3):
Viroid V0$+$V1$+$V2 produces \emph{more} alive entities at step~500 than
V0$+$V1 across all four harshness levels.
\emph{Rationale}: V2's regulator reduces overconsumption waste, providing a
survival advantage independent of V3 metabolism.

\subsection{Statistical Analysis}

Mann-Whitney U (two-tailed, $\alpha=0.05$), Cliff's~$\delta$ with 2000-resample
bootstrap 95\%~CI \citep{cliff_1993_dominance}, and Jonckheere-Terpstra trend
test for H4 and H7.
All $p$-values are Holm-Bonferroni corrected across the 32-test family
\citep{holm_1979_simple}.
Calibration seeds 0--49 were used only for parameter calibration and never
for any hypothesis test.
Any analysis not in the pre-registration is explicitly labelled
\emph{Exploratory} in the text.

% ============================================================================
\section{Results}
\label{sec:results}

\subsection{Phase Diagrams}
\label{sec:results-phase}

Figure~\ref{fig:phase} shows phase diagrams for all three archetypes.
Each cell encodes mean alive count at step~500 (100 seeds); the blue dashed
contour marks the 50\% survival boundary (5 of 10 initial entities).

\begin{figure*}[t]
\centering
\figifexists{figures/fig_semi_life_phase_diagram.pdf}{\textwidth}{}{phase diagram}
\caption{Phase diagrams: capability level $\times$ resource harshness
  $\rightarrow$ mean alive count at step~500 ($n\!=\!100$).
  YlOrRd heat-map with mean $\pm$ SD annotations; blue dashed contour =
  50\% survival boundary (5 entities).
  Left: Viroid (V0 through V0..V5).
  Centre: Virus (V0+V1 baseline through V0+V1+V2+V3).
  Right: ProtoOrganelle (V1+V2+V3 baseline vs.\ V0+V1+V2+V3 liberated).}
\label{fig:phase}
\end{figure*}

The Viroid panel reveals a non-monotonic capability ladder.
V0-only Viroid achieves mean alive counts of 93.0 (rich) and 38.1 (medium),
but adding V1 (boundary maintenance) \emph{reduces} survival to 33.1 and 17.6
respectively---boundary repair costs outweigh protective benefit.
V2 (homeostasis) adds negligible benefit (32.2 and 17.3).
V3 (metabolism) produces a dramatic recovery: 199.1 (rich) and 53.4 (medium),
exceeding even the V0-only baseline.
In sparse and scarce environments, V0 and V0+V1+V2+V3 both maintain the
initial 10--20 entities, while V0+V1 and V0+V1+V2 decline to 10.
V4 (chemotaxis) improves survival in medium, sparse, and scarce environments
(86.0, 35.9, 24.8 mean alive) but reduces it in rich (152.2 vs.\ V3's 199.1)
due to unnecessary movement cost.
V5 (lifecycle) recovers the rich-environment loss (204.7) through dormancy
but hurts in sparse/scarce environments where dormancy suppresses replication.

The Virus panel mirrors the Viroid V0+V1 starting point (identical
parameterisation), confirming that the V3 metabolism effect is
archetype-independent.
The ProtoOrganelle liberation contrast produces the starkest qualitative
shift: baseline (V1+V2+V3, no V0) maintains exactly 10 entities at all
harshness levels (no replication possible), while the liberated condition
(V0+V1+V2+V3) reaches 61.6 (rich) and 18.9 (medium).

\subsection{InternalizationIndex}
\label{sec:results-ii}

Figure~\ref{fig:ii} shows mean InternalizationIndex~(II) at step~500 for
Viroid across V-levels and harshness conditions.

\begin{figure}[t]
\centering
\figifexists{figures/fig_semi_life_internalization.pdf}{0.9\columnwidth}{}{II vs capability}
\caption{Multi-channel InternalizationIndex at step~500 (Viroid, $n\!=\!100$).
  Left: Per-channel II contributions as stacked bars (rich harshness);
  colours denote Energy (V3, orange), Regulation (V2, blue), Behaviour
  (V4, green), and Lifecycle (V5, pink).
  Right: Composite II (mean $\pm$ 95\% CI) across harshness levels.
  II rises gradually across the ladder as each V-level activates a new channel,
  rather than showing a step-function jump at V3.}
\label{fig:ii}
\end{figure}

With multi-channel II, the picture changes qualitatively from the single-channel
version.
V0-only yields $\mathrm{II}=0$ ($|C|=0$, no active channels).
V0+V1 still yields $\mathrm{II}=0$ (V1 does not contribute a channel).
V0+V1+V2 activates the regulation channel ($\mathrm{II}_R \geq 0$): $\mathrm{II}_R > 0$
when uptake exceeds the optimal rate (i.e., the overconsumption penalty fires),
which depends on resource availability and entity density.
In resource-poor environments where uptake rarely exceeds the optimal threshold,
$\mathrm{II}_R$ may remain near zero even with V2 active.
When $\mathrm{II}_R > 0$, this produces the first non-zero composite II---a key
difference from the prior single-channel formulation where V0--V2 all had
$\mathrm{II}=0$.
V3 addition activates the energy channel ($\mathrm{II}_E$), producing a larger
jump in composite II.
V4 adds the behaviour channel and V5 adds lifecycle, each incrementally raising
the composite.
Per-channel values and 95\% confidence intervals are shown in
Figure~\ref{fig:ii}.
In rich environments: V0+V1+V2 yields $\mathrm{II}_R=0.47$
(composite 0.23, one active channel).
V0+V1+V2+V3 yields $\mathrm{II}_E=0.62$, $\mathrm{II}_R=0.47$
(composite 0.71; higher than the channel mean because entities whose
regulation channel is inactive contribute only $\mathrm{II}_E$ to their
per-entity composite).
V0..V4 adds $\mathrm{II}_B=0.08$;
V0..V5 adds $\mathrm{II}_L=0.01$.
ProtoOrganelle baseline (V1+V2+V3, no V0) shows $\mathrm{II}>0$ because the
energy and regulation channels are active even without replication,
confirming that II measures internal functional organisation independently
of reproductive success.

\subsection{Pre-registered Hypothesis Tests}
\label{sec:results-stats}

Table~\ref{tab:stats} reports the pre-registered results.
All comparisons use $n\!=\!100$ test seeds and Holm-Bonferroni corrected
$p$-values across the 32-test family.

\begin{table}[t]
\centering
\footnotesize
\caption{Pre-registered hypothesis test results.
  $U$: Mann-Whitney statistic; $p_\text{corr}$: Holm-Bonferroni corrected;
  $\delta$: Cliff's~$\delta$ with 95\% CI in brackets;
  Dir.\ = pre-registered direction confirmed (\checkmark) or not (\texttimes).
  H4, H7 use Jonckheere-Terpstra; $\delta$ not applicable (---).}
\label{tab:stats}
\resizebox{\columnwidth}{!}{%
\begin{tabular}{llrrlrl}
\toprule
H & Harshness & $U$ & $p_\text{corr}$ & Sig. & $\delta$ [95\% CI] & Dir. \\
\midrule
\multicolumn{7}{l}{\emph{H1: Viroid V0 vs V0+V1 (alive, environment-dependent tradeoff)}} \\
H1 & rich   & 9\,981 & $<\!10^{-33}$ & *** & 1.00 [0.99, 1.00] & \checkmark \\
H1 & medium & 6\,473 & 0.002  & **  & 0.29 [0.15, 0.44] & \checkmark \\
H1 & sparse & 4\,950 & 1.000  &     & $-$0.01 [$-$0.03, 0.00] & \texttimes \\
H1 & scarce & 4\,950 & 1.000  &     & $-$0.01 [$-$0.03, 0.00] & \texttimes \\
\midrule
\multicolumn{7}{l}{\emph{H2: Viroid V0+V1+V2+V3 vs V0+V1+V2 (alive, more with V3)}} \\
H2 & rich   & 10\,000 & $<\!10^{-33}$ & *** & 1.00 [1.00, 1.00] & \checkmark \\
H2 & medium & 10\,000 & $<\!10^{-33}$ & *** & 1.00 [1.00, 1.00] & \checkmark \\
H2 & sparse & 10\,000 & $<\!10^{-43}$ & *** & 1.00 [1.00, 1.00] & \checkmark \\
H2 & scarce & 10\,000 & $<\!10^{-43}$ & *** & 1.00 [1.00, 1.00] & \checkmark \\
\midrule
\multicolumn{7}{l}{\emph{H3: ProtoOrganelle liberated vs baseline (total\_replications)}} \\
H3 & rich   & 10\,000 & $<\!10^{-37}$ & *** & 1.00 [1.00, 1.00] & \checkmark \\
H3 & medium & 10\,000 & $<\!10^{-38}$ & *** & 1.00 [1.00, 1.00] & \checkmark \\
H3 & sparse & 5\,000  & 1.000         &     & 0.00 [0.00, 0.00] & \texttimes \\
H3 & scarce & 5\,000  & 1.000         &     & 0.00 [0.00, 0.00] & \texttimes \\
\midrule
\multicolumn{7}{l}{\emph{H4: Viroid V0$\rightarrow$V3 monotonic trend (JT)}} \\
H4 & rich   & 20\,720 & $<\!10^{-12}$ & *** & --- & \checkmark \\
H4 & medium & 17\,937 & $<\!10^{-100}$ & *** & --- & \checkmark \\
H4 & sparse & 14\,900 & $<\!10^{-100}$ & *** & --- & \checkmark \\
H4 & scarce & 14\,900 & $<\!10^{-100}$ & *** & --- & \checkmark \\
\midrule
\multicolumn{7}{l}{\emph{H5: Viroid V0..V4 vs V0..V3 (alive, more with V4)}} \\
H5 & rich   & 96     & $<\!10^{-32}$ & *** & $-$0.98 [$-$1.00, $-$0.95] & \texttimes \\
H5 & medium & 9\,990 & $<\!10^{-33}$ & *** & 1.00 [0.99, 1.00] & \checkmark \\
H5 & sparse & 10\,000 & $<\!10^{-37}$ & *** & 1.00 [1.00, 1.00] & \checkmark \\
H5 & scarce & 9\,950 & $<\!10^{-37}$ & *** & 0.99 [0.97, 1.00] & \checkmark \\
\midrule
\multicolumn{7}{l}{\emph{H6: Viroid V0..V5 vs V0..V4 (alive, more with V5)}} \\
H6 & rich   & 9\,995 & $<\!10^{-33}$ & *** & 1.00 [1.00, 1.00] & \checkmark \\
H6 & medium & 1\,404 & $<\!10^{-17}$ & *** & $-$0.72 [$-$0.81, $-$0.61] & \texttimes \\
H6 & sparse & 0      & $<\!10^{-33}$ & *** & $-$1.00 [$-$1.00, $-$1.00] & \texttimes \\
H6 & scarce & 50     & $<\!10^{-37}$ & *** & $-$0.99 [$-$1.00, $-$0.97] & \texttimes \\
\midrule
\multicolumn{7}{l}{\emph{H7: Viroid V0$\rightarrow$V5 monotonic trend (JT)}} \\
H7 & rich   & 36\,893 & $<\!10^{-100}$ & *** & --- & \checkmark \\
H7 & medium & 27\,131 & $<\!10^{-100}$ & *** & --- & \checkmark \\
H7 & sparse & 25\,750 & $<\!10^{-100}$ & *** & --- & \checkmark \\
H7 & scarce & 29\,900 & $<\!10^{-100}$ & *** & --- & \checkmark \\
\midrule
\multicolumn{7}{l}{\emph{H8: Viroid V0+V1+V2 vs V0+V1 (alive, more with V2; Amendment~3)}} \\
H8 & rich   & 9\,152 & $<\!10^{-23}$ & *** & 0.83 [0.75, 0.90] & \checkmark \\
H8 & medium & 4\,995 & 1.000         &     & 0.00 [$-$0.15, 0.16] & \texttimes \\
H8 & sparse & 5\,000 & 1.000         &     & 0.00 [0.00, 0.00] & \texttimes \\
H8 & scarce & 5\,000 & 1.000         &     & 0.00 [0.00, 0.00] & \texttimes \\
\bottomrule
\end{tabular}%
}
\end{table}

\textbf{H1} (boundary cost--benefit tradeoff) reveals the predicted
environment-dependent tradeoff.
In rich environments, V1 significantly reduces survival ($\delta=1.00$,
$p_\text{corr}<10^{-33}$): mean alive drops from 33.9 (V0) to 21.6 (V0+V1),
because boundary maintenance cost exceeds the leakage/damage protection
benefit when resources are abundant.
In medium, the effect is weaker but still significant ($\delta=0.29$,
$p_\text{corr}=0.002$): 17.7~$\to$~16.9.
In sparse and scarce, V1 has \emph{no significant effect} ($\delta=-0.01$,
$p=1.0$): boundary protection exactly offsets its cost, yielding near-identical
populations ($\sim$10.0 alive).
This environment-dependent pattern---V1 hurts in rich but is neutral in
harsh---replaces the prior model's universal cost finding and demonstrates
a genuine tradeoff between protection and overhead.

\textbf{H2} (metabolism buffering) is confirmed universally ($\delta=1.00$,
all $p_\text{corr}<10^{-33}$).
V3 addition transforms survival: in rich environments, mean alive rises from
25.1 (V0+V1+V2) to 190.0 (V0+V1+V2+V3)---a 7.6$\times$ increase.
In medium, the jump is 17.0~$\to$~50.2 ($+195\%$).

\textbf{H3} (liberation) is confirmed in rich ($\delta=1.00$,
$p_\text{corr}<10^{-37}$) and medium ($\delta=1.00$, $p_\text{corr}<10^{-38}$),
where liberated ProtoOrganelle reaches 60.2 and 18.7 mean alive respectively.
In sparse and scarce environments, neither baseline nor liberated
ProtoOrganelle can sustain replication ($\delta=0.00$, $p=1.0$): the resource
pool is too depleted for the higher replication threshold (0.80) of this
archetype.

\textbf{H4} (monotonic trend) is now significant at \emph{all four}
harshness levels ($p_\text{corr}<10^{-12}$ in rich, $<10^{-100}$ elsewhere).
The revised V1/V2 mechanics produce a shallower cost valley: the alive
pattern in rich ($33.9 \to 21.6 \to 25.1 \to 190.0$) still shows a dip at
V1 before V3 recovery, but the V2 partial recovery (H8) creates a less
severe intermediate trough.
In sparse/scarce, the V0--V3 populations are nearly flat ($10 \to 10 \to 10
\to 20$) with V3 producing a consistent doubling, giving the JT test a
clear monotonic signal.

\textbf{H5} (chemotaxis benefit) is confirmed in medium ($\delta=1.00$),
sparse ($\delta=1.00$), and scarce ($\delta=0.99$), but \emph{reversed} in
rich ($\delta=-0.98$, $p_\text{corr}<10^{-32}$).
V4 chemotaxis improves survival where resources are heterogeneous:
mean alive rises from 50.2 to 77.5 in medium ($+54\%$) and from 20.0 to 33.4
in sparse ($+67\%$).
However, in resource-rich environments where resources are uniformly abundant,
the per-step movement cost ($0.01 \times |\Delta\mathbf{v}|$) reduces
survival from 190.0 to 139.7 ($-26\%$).
This reveals a \emph{second cost valley}: behavioural capabilities incur
overhead that is only justified when the environment rewards directed movement.

\textbf{H6} (lifecycle staging) is confirmed only in rich ($\delta=1.00$,
$p_\text{corr}<10^{-33}$), where V5 dormancy enables energy conservation
during transient low-resource periods, pushing mean alive from 139.7 to 182.6
($+31\%$).
In medium ($\delta=-0.72$), sparse ($\delta=-1.00$), and scarce
($\delta=-0.99$), V5 \emph{reduces} survival: dormancy phases suppress
replication windows, and dispersal's elevated energy decay ($\times 1.5$) is
counterproductive when resources cannot sustain the cost.
The environment-dependent reversal of H6 is the strongest qualitative finding
beyond the original V0--V3 ladder.

\textbf{H7} (full monotonic trend V0--V5) is significant at all four
harshness levels ($p_\text{corr}<10^{-100}$, Jonckheere-Terpstra).
Despite the non-monotonic pairwise reversals in H5 and H6, the overall
six-level trend V0$\to$V5 is robustly upward: the V3 metabolism recovery
dominates the signal, and V4/V5 additions, while environment-dependent,
do not reverse the aggregate trajectory.

\textbf{H8} (V2 overconsumption regulation, Amendment~3) is confirmed in
rich ($\delta=0.83$, $p_\text{corr}<10^{-23}$): mean alive rises from
21.6 (V0+V1) to 25.1 (V0+V1+V2), a $+16\%$ increase.
V2's regulation of uptake reduces overconsumption waste, providing a
measurable survival advantage.
In medium, sparse, and scarce, V2 has no significant effect
($\delta \approx 0$, $p=1.0$): overconsumption rarely occurs when
resources are scarce, so the regulation benefit does not materialise.
The rich-only significance of H8 demonstrates that V2 provides a
\emph{genuine but environment-dependent} benefit, not an engineered win.

\subsection{Replication--Persistence Tradeoff}
\label{sec:results-tradeoff}

Figure~\ref{fig:tradeoff} shows mean total replications vs.\ mean alive count
at step~500 for all conditions.

\begin{figure}[t]
\centering
\figifexists{figures/fig_semi_life_tradeoffs.pdf}{0.9\columnwidth}{}{tradeoff scatter}
\caption{Replication rate vs.\ persistence tradeoff.
  Each point represents one archetype condition $\times$ harshness level
  (mean over 100 seeds).
  Colour: archetype (Viroid: orange; Virus: sky-blue; ProtoOrganelle: green).
  Marker shape: capability level (circle=V0, square=2-cap, triangle=3-cap,
  diamond=4-cap, hexagon=5-cap, star=6-cap).
  Transparency encodes harshness (bright=rich, dim=scarce).}
\label{fig:tradeoff}
\end{figure}

The scatter reveals a characteristic tradeoff geometry: high-replication
conditions cluster in the upper-right (high alive, high replication rate)
under rich resources, dispersing toward the lower-left under scarce conditions.
ProtoOrganelle baseline ($\mathrm{V1+V2+V3}$, no V0) occupies the far left
(near-zero replications), while the liberated condition shifts markedly to
the right.
Archetype differences (colour) are visible even within the same capability
level, reflecting the parameter differences in replication cost and boundary
stability shown in Table~\ref{tab:archetypes}.

\subsection{Shock Resilience}
\label{sec:results-shock}

Figure~\ref{fig:recovery} compares Viroid alive trajectories under periodic
resource shocks (cycle period~$=50$) with no-shock baseline at sparse harshness.

\begin{figure*}[t]
\centering
\figifexists{figures/fig_semi_life_recovery.pdf}{\textwidth}{}{shock recovery}
\caption{Recovery under periodic resource shocks (Viroid, sparse harshness,
  $n\!=\!100$).
  Left: shock period~$=50$ (rapid crash--recovery cycles).
  Centre: shock period~$=200$ (slow cycles).
  Right: no-shock baseline (same harshness level for comparison).
  Each line is one capability level; Okabe-Ito palette.}
\label{fig:recovery}
\end{figure*}

\emph{Exploratory}: Under rapid shocks (period~$=50$), V0+V1+V2+V3 entities
maintain higher mean alive counts (20.0) than V0-only (19.2), with a large
effect size (Cliff's~$\delta=0.69$, 95\%~CI $[0.60, 0.78]$).
Under slow shocks (period~$=200$), the effect persists ($\delta=0.66$,
$[0.56, 0.75]$), as shown in the centre panel of Figure~\ref{fig:recovery}.
The recovery time analysis shows that most conditions recover within a single
sampling interval (50 steps), suggesting that at the current temporal
resolution, recovery time differences are not reliably measurable.
This comparison is exploratory: shock resilience was not included in the
H1--H8 pre-registration.
The shock experiment uses only sparse harshness; extending to other harshness
levels would test whether the V3 buffering advantage is consistent across
resource regimes and is left as future work.

\subsection{Effect Size Robustness Check}
\label{sec:results-energy}

\emph{Exploratory}: The alive-count metric used in H1--H8 exhibits
ceiling/floor effects in some comparisons ($\delta=1.00$ with $[1.00,1.00]$~CI),
raising the concern that the metric is too coarse to distinguish
strong from overwhelming effects.
As a robustness check, we repeat the H1 and H2 comparisons using
\texttt{mean\_energy} at step~500---a continuous per-entity measure that
avoids the binary alive/dead dichotomy.
The H1 energy comparison at rich harshness yields $\delta=-0.61$
with CI $[-0.73, -0.47]$---\emph{reversed} relative to the alive-count
direction.
V1 entities have higher mean energy ($0.432 \pm 0.033$) than V0 entities
($0.402 \pm 0.017$), indicating that boundary protection benefits individual
energy budgets even though population size decreases.
The H2 energy comparison at rich yields $\delta=0.33$ with CI $[0.15, 0.49]$,
confirming that V3 benefits are detectable but not saturated on the continuous
metric.
In medium, sparse, and scarce, both comparisons remain at $\delta=1.00$.
These results demonstrate that the revised model produces a richer range of
effect sizes ($\delta$ from $-0.61$ to $1.00$) than the prior model's
universally saturated values.

% ============================================================================
\section{Discussion}
\label{sec:discussion}

\subsection{What the Capability Ladder Tells Us}

The phase diagrams (Figure~\ref{fig:phase}) operationalise the virus-to-life
transition as a measurable shift in the capability~$\times$~harshness survival
boundary.
With the revised model (V1 leakage/damage protection, V2 overconsumption
regulation), the ladder reveals environment-dependent tradeoffs rather than
a universal cost valley.
V1 boundary maintenance shows a genuine cost--benefit tradeoff: it reduces
survival in resource-rich environments (where leakage and damage pressure
are low relative to repair costs) but provides net protection in harsh
environments (where stochastic damage is proportionally more costly).
V2 homeostasis mitigates overconsumption waste, providing measurable benefit
independent of V3 metabolism (H8).
V3 metabolism remains the largest single survival jump, consistent with
the biological intuition that internal resource buffering is the key
enabling capability.
This ``cost valley'' at V1/V2---followed by V3 recovery---is now shallower
and environment-dependent, mirroring the tradeoff structure expected from
real virus-to-cell transitions where capsid assembly is costly but provides
environmental protection.

Beyond V3, the ladder reveals a second, subtler cost-benefit structure.
V4 chemotaxis (H5) improves survival in medium through scarce environments
where resource gradients carry useful information (mean alive rises 61--80\%
over V3) but \emph{reduces} survival in rich environments ($-24\%$) where
uniform abundance makes movement wasteful.
V5 lifecycle staging (H6) shows the opposite environment dependence: dormancy
and dispersal pay off only in rich environments ($+34\%$ over V4), while in
sparse and scarce conditions the dormancy phases suppress replication windows
and dispersal's elevated energy cost ($\times 1.5$) is counterproductive.
These environment-dependent reversals suggest that ``more capabilities''
is not universally beneficial---each addition is only justified when
the environment rewards the specific function it provides.
Despite these pairwise reversals, the full V0$\to$V5 trend (H7, JT test)
is significant at all harshness levels ($p<10^{-100}$): the V3 metabolic
recovery dominates the aggregate signal.

The H3 ProtoOrganelle liberation result is environment-dependent:
in resource-rich conditions, adding V0 replication to a metabolically complete
entity (V1+V2+V3) immediately produces a replicating population
($\delta=1.00$); in sparse/scarce
environments, the resource pool is too depleted for the ProtoOrganelle's higher
replication threshold~(0.80).
The liberation contrast is partially self-evident (0 vs.\ non-0 replications);
its value lies in demonstrating that metabolic infrastructure (V1+V2+V3) alone
is insufficient without reproductive capability, and that the environment
must be sufficiently rich to sustain the higher replication threshold.
This suggests that the ``liberation'' of proto-endosymbionts---gaining
independent reproduction---may require not just the right capability but also
sufficiently favourable environmental conditions, consistent with serial
endosymbiosis scenarios \citep{maturana_1980_autopoiesis}.

The H4 monotonic trend is significant at all four harshness levels.
The Jonckheere-Terpstra test detects an overall upward tendency dominated by
the V3 recovery; the non-monotonic intermediate dip means the trend should
\emph{not} be interpreted as strictly monotonic at every step.
Rather, the virus-to-life transition is better modelled as a
\emph{cost-then-benefit} trajectory than as a smooth monotonic improvement.
The same caveat applies to H7's full V0$\to$V5 trend.

\subsection{InternalizationIndex as a Life-Likeness Metric}

The multi-channel II (Figure~\ref{fig:ii}) addresses a limitation of the
single-channel formulation: previously, II was effectively a V3 on/off
switch (zero for V0--V2, positive only with V3).
The revised four-channel composite rises gradually: V2 activates the
regulation channel, V3 the energy channel, V4 the behaviour channel, and
V5 the lifecycle channel.
Each channel is independently interpretable---$\mathrm{II}_E$ measures
metabolic self-sufficiency, $\mathrm{II}_R$ measures regulatory efficiency,
$\mathrm{II}_B$ measures behavioural autonomy, and $\mathrm{II}_L$ measures
lifecycle self-control---enabling validation that the composite is not
driven by a single dominant channel.
The V3 energy channel remains the largest contributor, consistent with
metabolism being the most transformative capability, but V2 and V4
contribute meaningfully.
Crucially, the survival benefit of V3 (seen in H2 and Figure~\ref{fig:phase})
is measured from alive counts, not from II, confirming structural
independence.

\subsection{Weak ALife Stance}

All claims in this paper are functional analogies.
``Boundary maintenance'' is functionally analogous to capsid integrity: it
provides a per-step cost plus a protective effect that prevents death and
blocks premature replication.
It does not claim to be a capsid, nor to model capsid biochemistry.
The same applies to homeostasis (V2), metabolism~(V3), chemotaxis~(V4),
and staged lifecycle~(V5).
We take a \emph{weak ALife} position \citep{langton_1989_artificial}:
the computational system models life-like properties without claiming to
\emph{be} alive or to resolve the definition of life.

\subsection{Limitations}

\paragraph{Platform dependence.}
The background world provides a structured resource landscape and temporal
dynamics, but the specific organism population parameters (30 organisms,
25 agents each) influence resource availability patterns.
The qualitative findings (cost valley, V3 recovery) are expected to be
robust to platform details, but exact effect sizes may shift under
different background populations.
Future work should validate on alternative resource landscapes
(e.g., static gradients, procedurally generated environments).

\paragraph{Scale and duration.}
The 10-entity initialisation per run and 500-step duration limit
population dynamics; for larger-scale phase diagrams,
$n_\text{init}\!=\!50{-}100$ and $T\!=\!2000$ would reduce stochastic
extinction artefacts.
An exploratory robustness analysis at these scales is planned.

\paragraph{Entity interaction.}
SemiLife entities do not compete with organisms or with each other---the
resource field is shared but there is no direct interaction.
Introducing competition would add ecological realism (resource
depletion effects, niche differentiation) but confounds the clean
capability-ladder interpretation; we propose competition as a priority
extension.

\subsection{Future Directions}

A competition axis---placing multiple archetypes in the same world simultaneously
---would reveal which capability profiles dominate under natural selection.
Extending V4 with nonlinear policies (small neural networks) could test
whether complex sensing strategies emerge under selection pressure.
Allowing V5 stage-transition parameters to evolve would test whether
lifecycle timing self-optimises across harshness gradients.
Finally, scaling to larger populations ($n_\text{init}\!=\!50{-}100$) and
longer runs ($>$1000 steps) would probe whether the cost valley persists
at ecological timescales.

% ============================================================================
\section*{Data Availability}

All simulation code, experiment scripts, pre-registration, and analysis
pipelines are available in the project repository (URL withheld for
double-blind review).
Raw TSV data files and statistical output (JSON) will be archived on
Zenodo upon acceptance, with a persistent DOI linked from the
camera-ready manuscript.

% ============================================================================

\footnotesize
\bibliographystyle{apalike}
\bibliography{references}

\end{document}
